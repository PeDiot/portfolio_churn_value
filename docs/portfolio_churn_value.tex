% Options for packages loaded elsewhere
\PassOptionsToPackage{unicode}{hyperref}
\PassOptionsToPackage{hyphens}{url}
%
\documentclass[
]{book}
\title{Portfolio, Churn \& Customer Value}
\author{Hugo Cornet, Pierre-Emmanuel Diot, Guillaume Le Halper, Djawed Mancer}
\date{2022-03-17}

\usepackage{amsmath,amssymb}
\usepackage{lmodern}
\usepackage{iftex}
\ifPDFTeX
  \usepackage[T1]{fontenc}
  \usepackage[utf8]{inputenc}
  \usepackage{textcomp} % provide euro and other symbols
\else % if luatex or xetex
  \usepackage{unicode-math}
  \defaultfontfeatures{Scale=MatchLowercase}
  \defaultfontfeatures[\rmfamily]{Ligatures=TeX,Scale=1}
\fi
% Use upquote if available, for straight quotes in verbatim environments
\IfFileExists{upquote.sty}{\usepackage{upquote}}{}
\IfFileExists{microtype.sty}{% use microtype if available
  \usepackage[]{microtype}
  \UseMicrotypeSet[protrusion]{basicmath} % disable protrusion for tt fonts
}{}
\makeatletter
\@ifundefined{KOMAClassName}{% if non-KOMA class
  \IfFileExists{parskip.sty}{%
    \usepackage{parskip}
  }{% else
    \setlength{\parindent}{0pt}
    \setlength{\parskip}{6pt plus 2pt minus 1pt}}
}{% if KOMA class
  \KOMAoptions{parskip=half}}
\makeatother
\usepackage{xcolor}
\IfFileExists{xurl.sty}{\usepackage{xurl}}{} % add URL line breaks if available
\IfFileExists{bookmark.sty}{\usepackage{bookmark}}{\usepackage{hyperref}}
\hypersetup{
  pdftitle={Portfolio, Churn \& Customer Value},
  pdfauthor={Hugo Cornet, Pierre-Emmanuel Diot, Guillaume Le Halper, Djawed Mancer},
  hidelinks,
  pdfcreator={LaTeX via pandoc}}
\urlstyle{same} % disable monospaced font for URLs
\usepackage{longtable,booktabs,array}
\usepackage{calc} % for calculating minipage widths
% Correct order of tables after \paragraph or \subparagraph
\usepackage{etoolbox}
\makeatletter
\patchcmd\longtable{\par}{\if@noskipsec\mbox{}\fi\par}{}{}
\makeatother
% Allow footnotes in longtable head/foot
\IfFileExists{footnotehyper.sty}{\usepackage{footnotehyper}}{\usepackage{footnote}}
\makesavenoteenv{longtable}
\usepackage{graphicx}
\makeatletter
\def\maxwidth{\ifdim\Gin@nat@width>\linewidth\linewidth\else\Gin@nat@width\fi}
\def\maxheight{\ifdim\Gin@nat@height>\textheight\textheight\else\Gin@nat@height\fi}
\makeatother
% Scale images if necessary, so that they will not overflow the page
% margins by default, and it is still possible to overwrite the defaults
% using explicit options in \includegraphics[width, height, ...]{}
\setkeys{Gin}{width=\maxwidth,height=\maxheight,keepaspectratio}
% Set default figure placement to htbp
\makeatletter
\def\fps@figure{htbp}
\makeatother
\setlength{\emergencystretch}{3em} % prevent overfull lines
\providecommand{\tightlist}{%
  \setlength{\itemsep}{0pt}\setlength{\parskip}{0pt}}
\setcounter{secnumdepth}{5}
\usepackage{booktabs}
\usepackage{amsthm}
\makeatletter
\def\thm@space@setup{%
  \thm@preskip=8pt plus 2pt minus 4pt
  \thm@postskip=\thm@preskip
}
\makeatother
\usepackage{booktabs}
\usepackage{longtable}
\usepackage{array}
\usepackage{multirow}
\usepackage{wrapfig}
\usepackage{float}
\usepackage{colortbl}
\usepackage{pdflscape}
\usepackage{tabu}
\usepackage{threeparttable}
\usepackage{threeparttablex}
\usepackage[normalem]{ulem}
\usepackage{makecell}
\usepackage{xcolor}
\ifLuaTeX
  \usepackage{selnolig}  % disable illegal ligatures
\fi
\usepackage[]{natbib}
\bibliographystyle{apalike}

\begin{document}
\maketitle

{
\setcounter{tocdepth}{1}
\tableofcontents
}
\hypertarget{abstract}{%
\chapter*{Abstract}\label{abstract}}
\addcontentsline{toc}{chapter}{Abstract}

This paper is being realized as part of our last year in master's degree in economics. It aims at studying the firm's most valuable asset: its customers. To that end, we adopt a quantitative approach based on a mix of Econometrics and Data Science techniques with a threefold purpose:

\begin{itemize}
\tightlist
\item
  Model customer \emph{portfolio} as a set of customer segments;
\item
  Predict and analyze customer \emph{attrition};
\item
  Estimate customer portfolio's overall \emph{value}.
\end{itemize}

After having defined the subject's key concepts, we apply duration models and machine learning algorithms to a \href{https://www.kaggle.com/yeanzc/telco-customer-churn-ibm-dataset}{kaggle} dataset related to customers of a fictional telecommunications service provider (TSP).

\textbf{\emph{Keywords}}: \emph{customer portfolio management (CPM), churn, customer value, duration models, machine learning, telecom.}

\hypertarget{intro}{%
\chapter{Introduction}\label{intro}}

In a world in which the access to information is almost free or insignificant and where there is a real plurality of offers, churn analysis has become one of the key points a firm needs to focus on. Whoever says plurality of offers needs to introduce the term competition. Thereby, the latter is more and more fierce and cut-throat. Furthermore, switching costs have decreased significantly thanks to market regulation laws. For instance in France, when you switch TSP, the new provider pays you off cancellation fees. All of this being said, it is essential for firms to implement efficient strategies to enhance customer relationships. To that end, the development of both survival models and machine learning algorithms have enabled companies to really push-up their strategies in terms of customer \emph{portfolio} management, monitoring of \emph{attrition} and estimation of customer \emph{value}.

After careful consideration of the issues at stake, the following key steps are focused on:

\begin{itemize}
\tightlist
\item
  Segmentation of customer portfolio as firms generally tend to partition their \emph{portfolio} into multiple segments.
\item
  Estimation of customer lifetime and prediction of \emph{attrition}.
\item
  Measurement of customer \emph{value.}
\end{itemize}

In the following sections, the concepts of \emph{portfolio}, \emph{attrition} and customer \emph{value} are defined. Then, some pieces of literature review are provided. Before embarking on data analysis and modelling, we present the theoretical basis of the models used in the study. We finally introduce the dataset and implement the methodology with the aim of estimating the overall value related to a fictional customer portfolio of a telecommunications service provider.

\hypertarget{portfoliodef}{%
\section{How to define a customer portfolio?}\label{portfoliodef}}

The notion of \emph{portfolio} has greatly evolved before the firms' consumer base was considered as a \emph{portfolio}. In chapter \ref{literature} a part of the literature review depicts an evolution of the \emph{portfolio} management notion. A customer \emph{portfolio} can be defined as a set of customers divided into several segments (or clusters) based on similar attributes. These discriminant features can be both economic (willingness to pay, budget constraint, etc.) and sociological (gender, age, socio-professional category, etc.). The underlying objective of this segmentation is to optimally allocate the company's resources.

When dealing with customer \emph{portfolio} management (CPM), two dimensions can be considered. On the one hand, it can be assumed that a customer stays in the same segment throughout their life in the firm's \emph{portfolio}. On the other hand, a dynamic approach can be adopted as suggested by \citet{MANAGING_DYNAMICS_CUSTOMER_PORTFOLIO} on dynamics in customer portfolio. In their article, the authors question the static analysis by assuming that a customer can switch between segments. They explain that one of the firm's objectives is to convert less valuable customers into more valuable ones.

\hypertarget{attritiondef}{%
\section{What is attrition?}\label{attritiondef}}

Customer \emph{attrition} or churn occurs when a client discontinues using a service or a product offered by a firm. Churn analysis corresponds to both measurement and prediction of the \emph{attrition} rate in the customer base of the company. Evaluating \emph{attrition} depends on the type of relationship between the firm and its clients. When it is defined by a contract, the customer has to inform the firm about their service termination. In the telecom industry, a consumer is required to notify their TSP before going to a competing company. In an opposite way, the firm/client relationship can be non-contractual. In that case, the service termination does not need to be notified. \emph{Attrition} then becomes a latent variable and more advanced models are used to make forecasts.

\hypertarget{valuedef}{%
\section{What does customer ``value'' mean?}\label{valuedef}}

In customer \emph{portfolio} management, one client's \emph{value} is represented by the \textbf{customer lifetime value} (CLV). CLV is the present \emph{value} of all future purchases made by a customer over their lifetime in the firm's portfolio, taking into account the \emph{attrition} risk. CLV depends both on the purchase recency as well as on the purchasing rate and aims at identifying the most valuable customer groups. Formally, \citet{CUSTOMERS_ASSETS} define CLV for customer \(i\) as follows:

\begin{equation}
  CLV_i = \sum_{t=0}^{T} \frac{(p_t - c_t)r_{i,t}}{(1+a)^t} - AC_i
  \label{eq:clv}
\end{equation}

with,

\begin{itemize}
\tightlist
\item
  \(p_t\) the price paid by customer \(i\) at time \(t\)
\item
  \(c_t\) the marginal cost at time \(t\)
\item
  \(r_{i,t}\) the \textbf{probability that customer \(i\) be active} at time \(t\)
\item
  \(a\) the discount rate
\item
  \(AC_i\) the acquisition cost of customer \(i\)
\item
  \(T\) the duration of observation
\end{itemize}

An estimation of the portfolio's overall value can be calculated through \textbf{customer equity} (CE) which amounts to the sum of all the CLVs. Since CE appears to be a good proxy of the firm's value, the firm's profit-maximization program can be written as:

\begin{equation}
\begin{aligned}
\max_{\mathrm{p}} \quad & \textrm{CE} = \sum_{i=1}^{N} \sum_{t=0}^{T} \frac{(p_t - c_t)r_{i,t}}{(1+a)^t} - AC_i\\
\textrm{s.t.} \quad & r_{i,t} \in [0, 1]\\
  &p_t > c_t    \\
\end{aligned}
\label{eq:maxprofit}
\end{equation}

where \(\mathrm{p}\) is the vector of prices over all periods that the firm needs to optimize.

\hypertarget{literature}{%
\chapter{Literature Review}\label{literature}}

Now the concepts of \emph{portfolio}, \emph{attrition} and \emph{value} have been defined, it seems relevant to take a look at the literature on these notions. The literature review made in this chapter synthesises and analyses the available articles related to customer \emph{portfolio} modelling, \emph{attrition} analysis as well as customer \emph{value} estimation. The review combines concepts from Economics, Econometrics and Data Science.

\hypertarget{portfolio}{%
\section{On customer portfolio}\label{portfolio}}

\emph{Portfolio} management methods have been applied to an increasing number of areas over time. This term is originally used in finance by \citet{MARKOWITZ} with a view of managing equities. He develops a mathematical framework for assembling a portfolio of assets such that the expected return is maximized for a given level of risk. Markovitz's model is based on diversification which is the idea that owning different kinds of financial assets is less risky than owning only one type. His theory uses the variance of asset prices as a proxy for risk. Later in the 1970-80's, \emph{portfolio} models are incorporated into corporate \citep{WIND} and marketing \citep{DAY} strategies for profit-maximization via optimal resource allocation. Then, \citet{CAPON} provide insights on efficient management for \emph{portfolios} of technologies and study the complementarity between technological means mobilized by a firm. More recently, in the interest of improving relationships between the firm and its clients, the \emph{portfolio} modelling approach have focused on effective customer relationship management (CRM). The following figure depicts the evolution of \emph{portfolio} analysis through time.

\begin{figure}

{\centering \includegraphics[width=400pt,height=300pt]{./imgs/portfolio_timeline} 

}

\caption{A timeline on the concept of portfolio}\label{fig:timeline}
\end{figure}

In their article \citet{CPM_CRM} examine how a company can define the
value of customers and segment these customers into \emph{portfolios}. They explain how segmentation leads to better understanding of the relative importance of each customer to the company's total profit. The authors consider a \emph{portfolio} segmented into four groups of clients: \emph{platinum}, \emph{gold}, \emph{silver} and \emph{bronze} customers. The \emph{portfolio} segmentation is based both on the cost to serve a client as well as the latter's \emph{value} to the firm, as depicted by figure \ref{fig:cpmmat}.

\begin{figure}

{\centering \includegraphics[width=300pt,height=300pt]{./imgs/cpm_matrix} 

}

\caption{Csutomer Portfolio Management (CPM) Matrix}\label{fig:cpmmat}
\end{figure}

According to this repartition into four main groups, Thakur and Workman highlight three strategies the firm can launch in order to efficiently manage its \emph{portfolio}. \textbf{Retention} aims to induce \emph{platinum} customers into repeating their purchases as they have a large contribution to the firm's revenue. \textbf{Customer relationship development} can be used to encourage customers to advance and upgrade to a higher segment. Such a strategy can be efficient for customers with high preference for a certain product or those with potential to shift to higher margin products. Conversely, \textbf{customer elimination or filtering} strategies are set up by the firm to encourage bottom customers who cost more than they are worth to leave the \emph{portfolio}.

As said in the introduction, an interesting improvement of \emph{portfolio} analysis may be to add a temporal dimension to the models. \citet{MANAGING_DYNAMICS_CUSTOMER_PORTFOLIO} show that the dynamic approach minimizes the current bias of underestimating low-value clients and overestimating high-value ones.

\hypertarget{attrition}{%
\section{On attrition}\label{attrition}}

\emph{Attrition} or churn has become a buzzword these last years. Churn analysis can be seen as an economic problem for three main reasons. Firstly, customers are in some way the firm's more precious asset. Secondly, the firm's resources in terms of customer relationship management are limited, so an efficient allocation needs to be deployed. Thirdly churn being a risk the firm has to cope with, it leads to asymmetric information from the firm's side. With the development of advanced Econometrics and Data Science, several methods can be implemented in order to estimate churn.

On the one hand, survival models are helpful in measuring customer lifetime. In her thesis, \citet{SURV_METHODS_INSURANCE} applies duration analysis to model the behavior of customers from an insurance company with a threefold purpose:

\begin{itemize}
\tightlist
\item
  Identify factors influencing customer loyalty.
\item
  Estimate the remaining lifetime of a client who has subscribed to multiple policies and cancelled one of them.
\item
  Study the influence of covariates changing over time.
\end{itemize}

Her study is motivated by the importance of the insurer/policyholder relationship in a digitalized environment where the costs of searching for information are loawer and lower and the risk of \emph{attrition} consequently higher. The author develops a two-part methodology to address the study's problematic. She begins by solely selecting insureds with at least two policies. Then, she fits a logit model to predict whether a policyholder will cancel their policies at the same time (type 1) or sequentially (type 2). She finally applies duration models on type 2 clients to determine the remaining time until all their policies are cancelled.

On the other hand, machine learning classification algorithms can be used for churn detection as illustrated by the work of \citet{CHURN_INSURANCE}. Her objective is to develop a predictive model to detect customer churn in an insurance company while highlighting the key drivers of \emph{attrition}. The underlying goals of her research paper are both minimizing revenue loss caused by churn and boosting the firm's competitiveness. Using data on vehicle insurance policies, Bellani incorporates features on the policyholders, the vehicles, the insurance policies as well as marketing data to predict the churn indicator variable. After missing data imputation and dimensionality reduction, the author falls back on under-sampling to overcome the issue of unbalanced classes. There are indeed much more active than cancelled policies in the dataset. Her methodology works as follows:

\begin{itemize}
\tightlist
\item
  The set of active policies is divided into 7 groups equal in size to the number of cancelled policies.
\item
  For each group of active policies, classification models (logistic regression, random forest and neural network) are trained on a subset of the original dataset including all the cancelled policies as well as the concerned group of active ones.
\item
  For each model, the predictions are aggregated across the 7 subsets for the final prediction.
\item
  Model selection is made by the means of the Kappa performance metric.
\end{itemize}

Ultimately when a customer leaves the firm's portfolio, it may worth it to consider all possible outcomes for the reason he churned. For instance, a client might leave their telecom company because of a bad service quality, or because of too high a price. In this context, competing risk analysis can be introduced since its main interest is to determine the reason why the client churned. In their recent article, \citet{COMPETING_RISKS} try to predict both the likelihood of customer churn and the reasons for \emph{attrition} using customer service data from a Dutch TSP. They estimate duration and competing risk models. In the competing risk model, three possible output states are considered: Controllable risk, Uncontrollable risk and Unknown risk. Each type of risk is assumed independent from another which means a client cannot be at high risk for two risks simultaneously. Besides, the authors implement a Latent Dirichlet Allocation model (see \citet{LDA} for more details) to identify the main topics in a set of emails sent by customers to the service center. Six topics are discovered by the algorithm and each of them is then incorporated as explanatory variable into the models. These topic variables increase the performance of both standard duration models and competing risk models for Controllable and Unknown risks. According to Slof, Frasincar, and Matsiiako, \emph{``customers who churn due to the Controllable risk or due to the Unknown risk tend to call the customer service center with a specific problem, while customers who churn due to the Uncontrollable risk do not call the customer service center with a specific problem''}.

\hypertarget{value}{%
\section{On customer value}\label{value}}

In recent years, customer \emph{portfolio} management (CPM) has focused on optimizing clients' \emph{value} to the firm. The company's interest lies in knowing how much net benefit it can expect from a customer today. These expectations are then used to implement efficient marketing strategies to get the highest return on investment. To that end, two key metrics are estimated by firms: customer lifetime value (CLV) and customer equity (CE) (see part \ref{valuedef} in the introduction for definitions).

According to \citet{CLV_DEF}, CLV is a temporal variable defined as the revenue derived from a customer minus the cost to the firm for maintaining the relationship with this very customer. As shown by \citet{CLV_CONTEXT}, CLV modelling depends on the type of relationship a firm has with its clients. In a contractual relationship, customer defections are observed which means that longer lifetime means higher customer value. Conversely, when the relationship is non-contractual, uncertainty arises between the customer's purchase behavior and lifetime.

With the development of data collection tools, companies have lots of customer-level data (or customer transaction data) at their disposal to measure CLV \citep{CLV_NBD}. Consequently, different modelling approaches can be adopted in order to estimate customer \emph{value.}

Recency Frequency Monetary (RFM) models are considered the simplest strategy to measure CLV and customer loyalty \citep{CLV}. They aim at targeting specific marketing campaigns at specific groups of customers to improve response rates. RFM models consist in creating clusters of clients based on three variables:

\begin{itemize}
\tightlist
\item
  \emph{Recency} which is the time that has elapsed since customers' last activity with the firm.
\item
  \emph{Frequency} that is the number of times customers transacted with the brand in a particular period of time.
\item
  \emph{Monetary} that is to say the value of customers' prior purchases.
\end{itemize}

However, RFM models have a limited predictive power since they only predict clients' behavior for the next period.

In their article on CLV management, \citet{CLV_MEASUREMENT} draw the review of more advanced modelling techniques that can be implemented to estimate customers' \emph{value.} A popular method to estimate customer lifetime value is the negative binomial distribution (NBD) - Pareto \citep{CLV_NBD} which helps solving the lifetime uncertainty issue. The model takes past customer purchase behavior as input such as the number of purchases in a specific time window and the date of last transaction. Then the model outputs a repurchase probability as well as a transaction forecast for each individual. In Borle and Singh's research paper, a hierarchical bayesian model is implemented with a view to jointly predict customer's churn risk and spending pattern. Here, the main advantage of using a bayesian approach is to give priors on CLV's drivers. The study is based on data coming from a membership-based direct marketing company where firm/client relationships are non-contractual. In other words, the times of each customer joining the membership and terminating it are known once these events happen. Thus the implementation of a sophisticated estimation strategy is justified.

In our study, emphasize is placed on estimating the overall value of a customer \emph{portfolio}. The methodology we develop is based on a research paper written by our Econometrics teacher Alain Bousquet, whose goal is to provide tools for an efficient management of patent \emph{portfolios} \citep{BREVETS}. The main idea is to consider each patent as an asset with a related value which can generate income if this very patent is exploited. The author emphasizes the importance to focus on the \emph{portfolio}'s \textbf{variance} on top of its expected value. Specifically, he explains that the variability in the probability of success in the exploitation of patents leads to a decrease in the overall risk to which the \emph{portfolio} is exposed. This modelling approach can be transposed to customer \emph{portfolio} analysis with the customer's \emph{value} corresponding to the CLV and the probability of exploitation being the opposite of the risk of \emph{attrition}. In this context, CLV can be estimated either with techniques mentioned above or regression methods. The customer's risk of churn can be modelled with duration models or machine learning techniques as evoked in \ref{attrition}. With this econometric framework, it is expected that customer heterogeneity be a key factor in the total variance of the portfolio's \emph{value}.

\hypertarget{duration}{%
\chapter{Duration models}\label{duration}}

This chapter presents theoretical basis of the models that are used to model customer \emph{portfolios.} As customer lifetime in a \emph{portfolio} is usually represented by the time to churn, duration models are adapted to the data we have at our disposal. Thus, this part focuses on introducing standard survival techniques.

\hypertarget{definition}{%
\section{Definition}\label{definition}}

According to \citet{CAMERON_TRIVEDI}, duration models (also called survival models) aims at measuring the time spent in a certain state before transitioning to another state. In Econometrics,

\begin{itemize}
\tightlist
\item
  a \textbf{state} corresponds to the class in which an individual \(i\) is at time \(t\).
\item
  a \textbf{transition} is movement from one state to another.
\item
  a \textbf{duration} measures the time spent in a certain state and is also called a \textbf{spell} length.
\end{itemize}

Since measuring the time until the event is needed for multiple purposes, duration analysis is used in a variety of economic sectors as depicted by the following table.

\begin{longtable}[]{@{}cc@{}}
\toprule
Economic sector & Purpose \\
\midrule
\endhead
Macroeconomics & Length of unemployment spells \\
Insurance & Risk analysis to offer a segmented pricing \\
Engineering & Time until a device breaks down \\
Epidemiology & Survival time of a virus \\
\textbf{Churn analysis} & \textbf{Time until a customer leaves the portfolio} \\
\bottomrule
\end{longtable}

\hypertarget{censoring-and-truncation}{%
\section{Censoring and Truncation}\label{censoring-and-truncation}}

When dealing with survival data, some observations are usually \textbf{censored} meaning they are related to spells which are not completely observed. Duration data can also suffer from a selection bias which is called \textbf{truncation}.

\hypertarget{censoring-mechanisms}{%
\subsection{Censoring mechanisms}\label{censoring-mechanisms}}

\textbf{Left-censoring} occurs when the event of interest occurs before the beginning of the observation period. For example, an individual is included in a study of unemployment duration at \(t_0\). At that time he has already been unemployed for a period but he cannot recall exactly the duration of this period. If we observe that he finds a job again at \(t_1\), we can only deduce that the duration of unemployment is bigger than \(t_1-t_0\), this individual is consequently left-censored. Observation 2 on figure \ref{fig:censoring} is associated with a left-censored spell \citep{LIU_SCOR}.

A spell is considered \textbf{right-censored} when it is observed from time \(t_0\) until a censoring time \(t_c\) as illustrated by observation 4 on figure \ref{fig:censoring}. For instance, the lifetime related to a customer who has not churned at the end of the observation period is right-censored. Let us note \(X_i\) the duration of a complete spell and \(C_i\) the duration of a right-censored spell. We also note \(T_i\) the duration actually observed and \(\delta_i\) the censoring indicator such that \(\delta_i = 1\) if the spell is censored. Then \((t_1, \delta_1),\dots,(t_N, \delta_N)\) are the realizations of the following random variables:

\begin{equation}
  \begin{aligned}
  T_i & = \min(X_i, C_i) \\
  \delta_i & = \pmb{1}_{X_i > C_i}
  \end{aligned}
  \label{eq:censoring}
\end{equation}

\hypertarget{selection-bias}{%
\subsection{Selection bias}\label{selection-bias}}

Survival data suffers from a \textbf{selection bias} (or truncation) when only a sub-sample of the population of interest is studied. A customer entering the firm's \emph{portfolio} after the end of the study is said to be \textbf{right-truncated}, whereas a client who has left the \emph{portfolio} before the beginning of the study is considered \textbf{left-truncated}. Mathematically, a random variable \(X\) is truncated by a subset \(A \in \mathbb{R}^+\) if instead of \(\Omega(X)\), we solely observe \(\Omega(X)\bigcap A\). On figure \ref{fig:censoring}, the first and fifth observations suffers from a selection bias.

\begin{figure}

{\centering \includegraphics[width=500pt,height=250pt]{./imgs/censoring_and_truncation} 

}

\caption{Censored and truncated data}\label{fig:censoring}
\end{figure}

\hypertarget{probabilistic-concepts}{%
\section{Probabilistic concepts}\label{probabilistic-concepts}}

In survival analysis, the response variable denoted \(T\) is a time-to-event variable. Instead of estimating the expected failure time, survival models estimate the \textbf{survival} and \textbf{hazard rate} functions which depend on the realization of \(T\).

\hypertarget{survival-function}{%
\subsection{Survival function}\label{survival-function}}

The survival function \(S(t)\) represents the probability that the considered event occurs after time \(t\). For instance, \(S(t)\) can measure the probability that a given customer survives in the \emph{portfolio} at least until time \(t\). Mathematically, the survival function is defined as:
\begin{equation}
  S(t) = P(T > t) = 1 - F(t)
  \label{eq:survfun}
\end{equation}

where \(F(t)\) is the cumulative distribution function.

\begin{figure}

{\centering \includegraphics[width=400pt,height=300pt]{./imgs/surv_fun_plot} 

}

\caption{Survival function $S_T(t)$ with $T \sim \mathcal{E} (1)$}\label{fig:survfunplot}
\end{figure}

\hypertarget{hazard-and-cumulative-hazard-functions}{%
\subsection{Hazard and Cumulative Hazard functions}\label{hazard-and-cumulative-hazard-functions}}

Another key concept in duration analysis is the hazard function \(\lambda(t)\) which approximates the probability that the event occurs at time \(t\). For instance, \(\lambda(t)\) can measure the probability that a given individual leaves the firm \emph{portfolio} at time \(t\). Formally, it is expressed as follows:
\begin{equation}
  \lambda(t) = \lim_{\Delta t \to 0} \frac{P\big[t \leq T < t + \Delta t | T \geq t \big]}{\Delta t}
  \label{eq:hazfun}
\end{equation}

Using the Bayes formula, equation \eqref{eq:hazfun} can also be written as (see proof \eqref{eq:hazfunproof} in the appendix):
\begin{equation}
  \lambda(t) = \frac{-\text{d} \ln \big(S(t)\big)}{\text{d} t}
  \label{eq:hazfunbis}
\end{equation}

Finally, integrating the instantaneous hazard function gives the cumulative hazard function which can be more precisely estimated than the hazard function \citep{CAMERON_TRIVEDI} and is defined as:

\begin{equation}
  \Lambda (t) = \int_{0}^{t} \lambda(s) \text{d}s = - \ln \big(S(t)\big)
  \label{eq:cumhazfun}
\end{equation}

\begin{figure}

{\centering \includegraphics[width=400pt,height=300pt]{./imgs/cum_haz_plot} 

}

\caption{Cumulative Hazard function $\Lambda_T(t)$ with $T \sim \mathcal{E} (1)$}\label{fig:cumhazplot}
\end{figure}

Thus, \textbf{the hazard, survival and cumulative hazard functions} are three mathematical functions which describe the same distribution.

\hypertarget{nonparam}{%
\section{Nonparametric models}\label{nonparam}}

When dealing with duration data, these methods are helpful to have a general overview of the raw (or unconditional) hazard. Nonparametric models are rather used for data description than prediction. No explanatory variable is included in these models except for treatment variables such as the type of contract a customer has subscribed.

\hypertarget{notations}{%
\subsection{Notations}\label{notations}}

Let us consider a sample with \(N\) observations with \(k\) ordered discrete failure times (e.g.~a failure can be a churn event), such that \(\forall j \in [\![1; k]\!]\) :

\begin{itemize}
\tightlist
\item
  \(t_j\) the \(j^{\text{th}}\) discrete failure time,
\item
  \(d_j\) the number of spells terminating at \(t_j\),
\item
  \(m_j\) the number of right-censored spells in the interval \([t_j, t_{j+1}]\),
\item
  \(r_j\) the number of exposed durations right before time \(t_j\) i.e.~at time \(t_j^{-}\), such that:
  \begin{equation}
  r_j = (d_j + m_j) + \dots + (d_k + m_k) = \sum_{l|l \geq j} (d_l + m_l)
  \label{eq:exposed}
  \end{equation}
\end{itemize}

\hypertarget{hazard-function-estimator}{%
\subsection{Hazard function estimator}\label{hazard-function-estimator}}

As the instantaneous hazard at time \(t_j\) is defined as \(\lambda_j = P[T=t_j|T\geq t_j]\), a trivial estimator of \(\lambda_j\) is obtained by dividing the number of durations for which the event is realized at \(t_j\) by the total number of exposed durations at time \(t_j^{-}\). Formally, it is expressed as:

\begin{equation}
  \hat{\lambda}_j = \frac{d_j}{r_j}
  \label{eq:hazest}
\end{equation}

\hypertarget{kaplan-meier-estimator}{%
\subsection{Kaplan-Meier estimator}\label{kaplan-meier-estimator}}

Once the hazard function estimator computed, the discrete-time survivor function can be estimated using the Kaplan-Meier product-limit estimator. To estimate the survival at time \(t\), this estimator computes the joint probability that a spell stays in the same state until \(t\) (e.g.~remaining loyal to a firm until a certain time). This method is based on conditional probabilities and the survival function estimate is defined as:

\begin{equation}
  \hat{S}(t) = \Pi_{j|t_j \leq t} \big(1-\hat{\lambda}_j\big) = \Pi_{j|t_j \leq t}\frac{r_j - d_j}{r_j}
  \label{eq:kaplanmeier}
\end{equation}

When plotting the survival curve after having performed the Kaplan-Meier estimation, confidence bands are also added to the plot in order to reflect sampling variability \citep{CAMERON_TRIVEDI}. The confidence interval of the survival function \(\hat{S}(t)\) is derived from the estimate of the variance of \(S(t)\) which is obtained by the Grenwood estimate as in equation \eqref{eq:greenwood}.

\begin{equation}
  \widehat{\mathrm{V}}[\hat{S}(t)] = \hat{S}(t)^2 \sum_{j|t_j \leq t} \frac{d_j}{r_j(r_j-d_j)}
  \label{eq:greenwood}
\end{equation}

\hypertarget{nelson-aalen-estimator}{%
\subsection{Nelson-Aalen estimator}\label{nelson-aalen-estimator}}

The cumulative hazard function estimate is given by the Nelson-Aalen estimator which consists in summing up the hazard estimates for each discrete failure time.

\begin{equation}
  \hat{\Lambda}(t) = \sum_{j | t_j \leq t} \hat{\lambda}_{j} = \sum_{j | t_j \leq t} \frac{d_j}{r_j}
  \label{eq:nelsonaalen}
\end{equation}

Exponentiating \(\hat{\Lambda}(t)\), one can obtain a second estimate of the survival function (see proof \eqref{eq:linksurvcumhaz} in the appendix):

\begin{equation}
    \tilde{S}(t) = \exp \big( -\hat{\Lambda}(t) \big)
    \label{eq:survest}
\end{equation}

\hypertarget{parametric-models}{%
\section{Parametric models}\label{parametric-models}}

The nonparametric estimation is undoubtedly useful when it comes to have a general overview on the survival data. However, one may want to model the hazard and survivor functions with a functional form in which unknown parameters need to be optimized.

Parametric estimation has a twofold purpose that is to implement a robust model to estimate the risk that a specific event occurs while identifying the variables (or covariates) which best explain this risk.

When implementing parametric models, \(\lambda\), \(S\) and \(\Lambda\) are expressed based on the chosen parametric form. The instantaneous hazard function can either be constant or monotone.

In our study we assume that the explanatory variables are time-constant as we do not have dynamic data at our disposal. Thus, solely time-invariant duration models are presented.

\hypertarget{constant-hazard-exponential-model}{%
\subsection{Constant hazard (exponential model)}\label{constant-hazard-exponential-model}}

The exponential distribution models the time between events in a Poisson process and has the key property of being \emph{memoryless}. Let us note \(T\) a time-to-event variable such that \(T \sim \mathcal{E}(\theta)\) where \(\theta\) is the rate parameter. In this context, \emph{memorylessness} can be defined as follows:

\begin{equation}
  \mathbb{P}(T>t+s | T > t) = \mathbb{P}(T>s)
  \label{eq:memorylessness}
\end{equation}

\(\forall t \geq 0\ ,\ \theta > 0\) the density, hazard and survival functions can be expressed as:

\begin{equation}
  \begin{aligned}
    f_{\theta}(t) & = \theta e^{-\theta t} \\\\
    \lambda_{\theta}(t) & = \theta \\\\
    S_{\theta}(t) & = e^{-\theta t} \\\\
  \end{aligned}
  \label{eq:exponential}
\end{equation}

Thus, the exponential distribution is characterized by a \textbf{constant} hazard function which is a consequence of the \emph{memorylessness} property.

\hypertarget{monotone-hazard}{%
\subsection{Monotone hazard}\label{monotone-hazard}}

\hypertarget{weibull-model}{%
\subsubsection*{Weibull model}\label{weibull-model}}
\addcontentsline{toc}{subsubsection}{Weibull model}

The Weibull distribution is a less restrictive generalization of the exponential distribution defined by a shape parameter \(\nu\) and a scale parameter \(\theta\).

\(\forall t \geq 0\) and \(\nu,\ \theta > 0\) the density, hazard and survival functions can be expressed as:

\begin{equation}
  \begin{aligned}
      \lambda_{\nu,\theta}(t) & = \nu \bigg(\frac{1}{\theta}\bigg)^{\nu} t^{\nu - 1} \\\\
      S_{\nu, \theta}(t) & = \exp \Bigg( -\bigg(\frac{1}{\theta}\bigg)^{\nu} t\Bigg) 
  \end{aligned}
  \label{eq:weibull}
\end{equation}

The instantaneous hazard function \(\lambda_{\nu,\theta}\) is monotonic \textbf{decreasing} if \(\nu \in [0, 1]\). For instance, the \emph{attrition} risk may decrease as the customer's duration in the \emph{portfolio} increases. In this context, the client gets more and more loyal to the firm. If \(\nu=1\), the hazard rate is constant and \(T \sim \mathcal{E}(\theta)\). Converlsely, the hazard function is monotonic \textbf{increasing} if \(\nu > 1\). This can be the case when customers tend to continuously search for information on the firm's competitors, thus becoming more likely to churn as time goes by.

Figure \ref{fig:weibullplots} illustrates the hazard and survivor functions associated to a Weibull-distributed variable \(T\). The two curves' shape depend both on the shape (\(\nu\)) and scale (\(\theta\)) parameters. Some remarks can be made looking at the two plots. When \(\nu < 1\) the hazard function is decreasing meaning that the risk of the event occurring decreases as time goes by. When \(\nu > 1\) the hazard function is convex increasing which indicates that a marginal increase in time leads to an increase of over one unit in the the hazard function. The higher the shape parameter, the more increasing the hazard function. When \(\nu = \theta = 1\), it can be noted that the Weibull distribution corresponds to the exponential distribution (see figures \ref{fig:survfunplot} and \ref{fig:cumhazplot}).

\begin{figure}

{\centering \includegraphics[width=700pt,height=350pt]{./imgs/weibull_plots} 

}

\caption{Hazard and Survival functions with $T \sim \mathcal{W} (\nu, \theta)$}\label{fig:weibullplots}
\end{figure}

\hypertarget{other-models}{%
\subsubsection*{Other models}\label{other-models}}
\addcontentsline{toc}{subsubsection}{Other models}

Different probabilistic distributions can be chosen to model the hazard and survival functions related to a time-to-event variable with monotone hazard. The Gompertz model is usually used for mortality data in biostatistics. As for the gamma model, it depends both on the gamma and inverse-gamma distributions and is also based on shape and scale parameters.

\hypertarget{concave-and-convex-hazard}{%
\subsection{Concave and convex hazard}\label{concave-and-convex-hazard}}

When the hazard function does not evolve in a monotonic fashion, the distributions introduced above are limited. The generalized Weibull model appears to be a good choice to estimate phenomena with concave or convex hazards. It is based on three parameters: \(\nu\) (shape), \(\theta\) (scale) and \(\gamma\). When \(\gamma = 1\), the generalized Weibull becomes the Weibull distribution \(\mathcal{W}(\nu, \theta)\).

\hypertarget{semi-parametric-estimation}{%
\section{Semi-parametric estimation}\label{semi-parametric-estimation}}

\hypertarget{proportional-hazards-models}{%
\subsection{\texorpdfstring{\emph{Proportional Hazards} models}{Proportional Hazards models}}\label{proportional-hazards-models}}

Parametric models assume that the baseline (or raw) hazard follows a specific distribution. This assumption can be sometimes too restrictive and semi-parametric models can be more adapted to describe the duration data.

In \emph{proportional hazards} (PH) models, the instantaneous risk function is \textbf{proportional} to the baseline hazard \(\lambda_0 (t,\alpha)\) modulo a \textbf{scaling factor} depending on the covariates \(\phi(\pmb{\mathrm{x}}, \beta)\). These models allow to generalize the basic survival models to a survival regression model which permits to take individuals' heterogeneity into consideration \citep{RMS}. The general mathematical formulation is expressed as follows:

\begin{equation}
    \lambda(t|\pmb{\mathrm{x}}) = \lambda_0 (t,\alpha)  \phi(\pmb{\mathrm{x}}, \beta) 
    \label{eq:ph}
\end{equation}

Note that when the function form of \(\lambda_0 (t,\alpha)\) is known, we are in the case of parametric estimation. For instance, the exponential, Weibull and Gompertz models are PH models since their respective hazards are function of some covariates.

\hypertarget{what-does-proportional-hazards-mean}{%
\subsubsection*{\texorpdfstring{What does \emph{proportional hazards} mean?}{What does proportional hazards mean?}}\label{what-does-proportional-hazards-mean}}
\addcontentsline{toc}{subsubsection}{What does \emph{proportional hazards} mean?}

PH models are said to be proportional as the relative hazard ratio between two individuals \(i\) and \(k\) does not vary over time, such that:

\begin{equation}
  \frac{\lambda(t|\mathrm{x_i})}{\lambda(t|\mathrm{x_k})} = \frac{\phi(\mathrm{x_i}, \beta) }{\phi(\mathrm{x_k}, \beta)}
  \label{eq:prophazratio}
\end{equation}

The formulation stated in equation \eqref{eq:prophazratio} needs to be verified when one wants to fit a PH model to real-life data and is only valid in the case of time-constant covariates.

\hypertarget{marginal-effects}{%
\subsubsection*{Marginal effects}\label{marginal-effects}}
\addcontentsline{toc}{subsubsection}{Marginal effects}

In \emph{proportional hazards} models, the marginal effect of covariate \(x_p\) on the hazard function can be easily derived since this computation only requires knowledge on \(\beta\). As shown in \citet{CAMERON_TRIVEDI}, a one-unit increase in the \(p^{\text{th}}\) covariate leads to the following variation in the hazard function \emph{ceteris paribus}:

\begin{equation}
    \frac{\partial \lambda(t|\pmb{\mathrm{x}}, \beta)}{\partial x_p} = \lambda(t|\pmb{\mathrm{x}}, \beta) \frac{\partial \phi(\pmb{\mathrm{x}}, \beta) / \partial x_p}{\phi(\pmb{\mathrm{x}}, \beta) }
    \label{eq:meph}
\end{equation}

Thus the new hazard after variation of the \(p^{\text{th}}\) covariate is the original hazard times the effect of \(x_p\) on the model's regression part.

\hypertarget{partial-likelihood-estimation}{%
\subsubsection*{Partial likelihood estimation}\label{partial-likelihood-estimation}}
\addcontentsline{toc}{subsubsection}{Partial likelihood estimation}

The vector of parameters \(\beta\) related to the regression part of the PH model is estimated by partial likelihood maximization. The method's principle consists in only estimating the regression's parameters \(\beta\) by considering the baseline hazard \(\lambda_0\) as noise. If desired an estimate of the baseline hazard can be recovered after estimation of \(\beta\) using, for instance, the Nelson-Aalen estimator (see part \ref{nonparam}). Cox's intuition is that no information can be retrieved from the intervals during which no event has occurred and that it is conceivable that \(\lambda_0\) is null in these intervals. Thus, solely the set of moments when an event occurs are considered in the estimation method.

In order to derive the partial likelihood function, let us note \(t_j\) the \(j^{\text{th}}\) discrete failure time in an \(N\)-sample with \(j \in [\![1; k]\!]\), such that:

\begin{itemize}
\tightlist
\item
  \(t_1 < t_2 < \dots < t_k\),
\item
  \(D(t_j) = \{l: t_l = t_j\}\) is the set of spells completed at \(t_j\) with \(\#D(t_j) = d_j\),
\item
  \(R(t_j) = \{l: t_l \geq t_j\}\) is the set of spells at risk at \(t_j\).
\end{itemize}

The contribution of a spell in \(D(t_j)\) to the likelihood function equals the conditional probability that the spell ends at \(t_j\) given it is exposed at that specific time and can be written as (see \citet{CAMERON_TRIVEDI} and proof \eqref{eq:contribpartiallikproof} for more details):

\begin{equation}
  \mathbb{P}\big[T_j = t_j | R(t_j) \big] = \frac{\phi(\mathrm{x_j}, \beta)}{\sum_{l \in R(t_j)} \phi(\mathrm{x_l}, \beta)}
  \label{eq:contribpartiallik}
\end{equation}

Given \(k\) discrete failure times are considered and that for each of those there is a set \(D(t_j)\) of completed spells, Cox defines the partial likelihood function as the joint product of the probability expressed in \eqref{eq:contribpartiallik}, such that:

\begin{equation}
  \mathcal{L}_p = \Pi_{j=1}^{k} \ \frac{\Pi_{m \in D(tj)} \ \phi(\mathrm{x_j}, \beta)}{\Big[\sum_{l \in R(t_j)} \phi(\mathrm{x_l}, \beta)\Big]^{d_j}}
  \label{eq:partlik}
\end{equation}

The latter formulation of the partial likelihood function is explained in more details in proofs \eqref{eq:partlikproof} and \eqref{eq:partlikproofbis} in the appendix.

\hypertarget{coxph}{%
\subsection{Cox PH model}\label{coxph}}

The Cox \emph{proportional hazards} model is the most popular for the analysis of duration data. This model is said to be semi-parametric as it makes no assumption regarding the nature of the baseline hazard function \(\lambda_0(t)\). The parametric part only relies in the modelling of the effect of some covariates on the hazard function \(\lambda(t)\). The relationship between the vector of covariates and the log hazard is linear and the parameters can be estimated by maximizing the partial likelihood function. The Cox PH model solely assumes that predictors act multiplicatively on the hazard function. The model is formulated as in equation \eqref{eq:ph} with the exponential function as link between the hazard and the covariates i.e.~\(\lambda(t|\pmb{\mathrm{x}}) = \lambda_0 (t,\alpha) \text{e}^{\pmb{\mathrm{x'}} \beta}\).

\hypertarget{performance-metrics}{%
\section{Performance metrics}\label{performance-metrics}}

\hypertarget{concordance-index-c-index}{%
\subsection{Concordance index (C-index)}\label{concordance-index-c-index}}

C-index is a goodness of fit measure for models which produce risk scores. It is commonly used to evaluate risk models in survival analysis, where data may be censored.

Consider both the observations and prediction values of two instances \((y_1; \hat{y}_1)\) and \((y_2; \hat{y}_2)\). \(y_i\) and \(\hat{y}_i\) represent respectively the actual observation time and the predicted time. Mathematically, the C-index is defined as the probability to well predict the order of event occurring time for any pair of instances.

\begin{equation}
  c = \mathbb{P}\big(\hat{y}_1 > \hat{y}_2 | y_1 > y_2\big)
  \label{eq:cindex}
\end{equation}

Another way to write the C-index metric is to compute the ratio between concordant pairs and the total number of pairs. Consider individual \(i\) and let \(T\) be the time-to-event variable and \(\eta_i\) the risk score assigned to \(i\) by the model. We say that the pair \((i, j)\) is a concordant pair if \(\eta_i > \eta_j\) and \(T_i < T_j\), and it is a discordant pair if \(\eta_i > \eta_j\) and \(T_i > T_j\). If both \(T_i\) and \(T_j\) are censored, then this pair is not taken into account in the computation. If \(T_j\) is censored, then:

\begin{itemize}
\tightlist
\item
  If \(T_j < T_i\) the pair \((i, j)\) is not considered in the computation since the order cannot be determined.
\item
  If \(T_j > T_i\), the order can be determined and \((i, j)\) is concordant if \(\eta_i > \eta_j\), discordant otherwise.
\end{itemize}

Equation \eqref{eq:cindex} can then be rewritten as follows:

\begin{equation}
\begin{aligned}
c & = \frac{\# \text{concordant pairs}}{\# \text{concordant pairs} + \# \text{discordant pairs}} \\\\
c & = \frac{\sum_{i \neq j} \pmb{1}_{\eta_i < \eta_j} \pmb{1}_{T_i > T_j}d_j}{\sum_{i \neq j} \pmb{1}_{T_i > T_j}d_j}
\end{aligned}
\label{eq:cindex2}
\end{equation}

with \(d_j\) the event indicator variable.

The concordance index ranges between 0 and 1. A C-index below 0.5 indicates a very poor model. A C-index of 0.5 means that the model is rather a non-informative model making random predictions. A model with C-index 1 makes perfect prediction. Generally, a C-index higher than 0.7 indicates a good performance.

\hypertarget{brier-score}{%
\subsection{Brier score}\label{brier-score}}

The Brier score is another statistical metric for evaluating duration models' performance and is defined as the mean squared error between the estimated survival probability and the observed survival at time \(t\):

\begin{equation}
  BS(t) = \frac{1}{N} \sum_{i=1}^{N} \Big(\pmb{1}_{\{t_i>t\}} - \hat{S}(t|\mathrm{x}_i) \Big)^2
  \label{eq:brier}
\end{equation}

The Cox \emph{proportional hazards} model is the most popular for the analysis of duration data. This model is said to be semi-parametric as it makes no assumption regarding the nature of the baseline hazard function \(\lambda_0(t)\). The parametric part only relies in the modelling of the effect of some covariates on the hazard function \(\lambda(t)\). The relationship between the vector of covariates and the log hazard is linear and the parameters can be estimated by maximizing the partial likelihood function. The Cox PH model solely assumes that predictors act multiplicatively on the hazard function. The model is formulated as in equation \eqref{eq:ph} with the exponential function as link between the hazard and the covariates i.e.~\(\lambda(t|\pmb{\mathrm{x}}) = \lambda_0 (t,\alpha) \text{e}^{\pmb{\mathrm{x'}} \beta}\).

\hypertarget{data}{%
\chapter{Data}\label{data}}

In this chapter, we introduce the \href{https://www.kaggle.com/yeanzc/telco-customer-churn-ibm-dataset}{kaggle} dataset related to customers of a fictional telecommunications service provider (TSP). In this duration dataset, the \texttt{Churn\_Value} status variable indicates whether the customer left the firm's \emph{portfolio} within the last month while the \texttt{Tenure\_Months} variable stands for the duration actually observed. Besides customer \emph{value} can be approximated by the \texttt{CLTV} variable.

\hypertarget{general-overview}{%
\section{General Overview}\label{general-overview}}

The data set used in this study contains 29 variables and 7032 customers from a telecom firm. For each client, the data includes:

\begin{itemize}
\item
  \textbf{Demographic} information: \texttt{CustomerID}, \texttt{City}, \texttt{Zip\_Code}, \texttt{Latitude}, \texttt{Longitude}, \texttt{Gender}, \texttt{Senior\_Citizen}, \texttt{Partner} and \texttt{Dependents}.
\item
  \textbf{Customer account} information: \texttt{Tenure\_Months}, \texttt{Contract}, \texttt{Paperless\_Billing}, \texttt{Payment\_Method}, \texttt{Monthly\_Charges}, \texttt{Total\_Charges}, \texttt{Churn\_Label}, \texttt{Churn\_Value}, \texttt{Churn\_Score}, \texttt{CLTV}, \texttt{Churn\_Reason}.
\item
  \textbf{Services} information: \texttt{Phone\_Service}, \texttt{Multiple\_Lines}, \texttt{Internet\_Service}, \texttt{Online\_Security}, \texttt{Online\_Backup}, \texttt{Device\_Protection}, \texttt{Tech\_Support}, \texttt{Streaming\_TV}, \texttt{Streaming\_Movies}.
\end{itemize}

\begin{table}[H]

\caption{\label{tab:dataoverview}Interesting variables for the 5 first customers in the data set}
\centering
\begin{tabular}[t]{rrlrrr}
\toprule
CustomerID & Monthly\_Charges & Internet\_Service & Tenure\_Months & Churn\_Value & CLTV\\
\midrule
\cellcolor{gray!6}{1} & \cellcolor{gray!6}{53.85} & \cellcolor{gray!6}{DSL} & \cellcolor{gray!6}{2} & \cellcolor{gray!6}{1} & \cellcolor{gray!6}{3239}\\
2 & 70.70 & Fiber optic & 2 & 1 & 2701\\
\cellcolor{gray!6}{3} & \cellcolor{gray!6}{99.65} & \cellcolor{gray!6}{Fiber optic} & \cellcolor{gray!6}{8} & \cellcolor{gray!6}{1} & \cellcolor{gray!6}{5372}\\
4 & 104.80 & Fiber optic & 28 & 1 & 5003\\
\cellcolor{gray!6}{5} & \cellcolor{gray!6}{103.70} & \cellcolor{gray!6}{Fiber optic} & \cellcolor{gray!6}{49} & \cellcolor{gray!6}{1} & \cellcolor{gray!6}{5340}\\
\addlinespace
6 & 55.20 & DSL & 10 & 1 & 5925\\
\bottomrule
\end{tabular}
\end{table}

As shown by table \ref{tab:dataoverview}, the \texttt{Churn\_Value} status variable indicates whether the customer left the firm's \emph{portfolio} within the last month and \texttt{Tenure\_Months} is the duration variable.

Since the purpose of our study relies in estimating the overall value of this fictional firm's \emph{portfolio}, two groups of target variables can be considered. On the one hand \texttt{Churn\_Value} and \texttt{Tenure\_Months} permit to determine whether a customer is active in the \emph{portfolio}. They are used as response variables in the survival models. On the other hand, the \texttt{CLTV} variable represents each customer's \emph{value} through measurement of customer lifetime value and is used as target in regression models.

\hypertarget{churndescstats}{%
\section{\texorpdfstring{\texttt{Churn\_Value} and \texttt{Tenure\_Months}}{Churn\_Value and Tenure\_Months}}\label{churndescstats}}

As the combination of these two features form the response variable in the duration models, a relevant approach to have an overall description of the risk of \emph{attrition} may be to draw the raw survival curves depending on treatment variables. Pearson's \(\chi^2\) tests are also performed so as to test the statistical relationships between the churn indicator variable and explanatory features. Pearson's \(\chi^2\) test determines whether there is a statistically significant difference between the expected frequencies and the observed frequencies in one or more categories of a contingency table. It is thus adapted to test whether two categorical variables are statistically independent. In this context, it appears interesting to use explanatory features related demographic, customer account and services subscribed as treatment variables when fitting the Kaplan-Meier estimator and implementing the tests.

\hypertarget{demographic-data}{%
\subsection*{Demographic data}\label{demographic-data}}
\addcontentsline{toc}{subsection}{Demographic data}

Table \ref{tab:chi2demographics} depicts the \(\chi^2\) tests' results performed on demographic variables. Given the \emph{p-values} are ranked in ascending order and given the lower the \emph{p value} the stronger link between two categorical variables, \texttt{Dependents} appears to be the most correlated feature with \texttt{Churn\_Label}. When comparing this result with the corresponding survival plot in figure \ref{fig:kmdemographics}, it can be noted that customers with dependants have a longer lifetime in the \emph{portfolio}. Conversely, \texttt{Gender} and \texttt{Churn\_Label} are statistically independent as stated by the high test's \emph{p value} (\(\approx .49\)).

\begin{table}[H]

\caption{\label{tab:chi2demographics}Independence $\chi^2$ test between churn and demographic variables}
\centering
\begin{tabular}[t]{lrrrl}
\toprule
  & Statistic & Df & Critical Value & p-value\\
\midrule
\cellcolor{gray!6}{Dependents} & \cellcolor{gray!6}{431.65} & \cellcolor{gray!6}{1} & \cellcolor{gray!6}{3.84} & \cellcolor{gray!6}{7.1e-96}\\
Senior\_Citizen & 158.44 & 1 & 3.84 & 2.5e-36\\
\cellcolor{gray!6}{Partner} & \cellcolor{gray!6}{157.50} & \cellcolor{gray!6}{1} & \cellcolor{gray!6}{3.84} & \cellcolor{gray!6}{4e-36}\\
Gender & 0.48 & 1 & 3.84 & 4.9e-01\\
\bottomrule
\end{tabular}
\end{table}

In section \ref{nonparam}, nonparametric estimation has been introduced focusing on two major estimators: Kaplan-Meier for survival function estimation and Nelson-Aalen for estimating the cumulative hazard function. In this part, it is decided to draw survival curves related to customer lifetime in the portfolio depending on different types of treatment variables. In the figure below, four main results can be highlighted \emph{ceteris paribus}:

\begin{itemize}
\tightlist
\item
  There seems to be no difference in terms of lifetime duration between men and women.
\item
  Customers with a partner appear to stay longer in the TSP's \emph{portfolio}.
\item
  Being a senior citizen tends to shorten customer lifetime.
\item
  As said before, customer with children or other dependents seem to be more loyal.
\end{itemize}

\begin{figure}

{\centering \includegraphics[width=50in]{./imgs/demographics_plot} 

}

\caption{Kaplan-Meier survival function depending on demographic information}\label{fig:kmdemographics}
\end{figure}

\hypertarget{data-on-services-subscribed}{%
\subsection*{Data on services subscribed}\label{data-on-services-subscribed}}
\addcontentsline{toc}{subsection}{Data on services subscribed}

When dealing with data on customers of a TSP, features related to services subscribed may be relevant to explain the estimated survival of these customers in the \emph{portfolio}.

Table \ref{tab:chi2services} presents results of \(\chi^2\) tests performed between \texttt{Churn\_Label} and each services information variable. As in the previous table, \emph{p-values} are ranked in ascending order. One can note that \texttt{Online\_Security} and \texttt{Tech\_Support} are the most linked to the churn indicator variable. However, variable carrying information on phone services are less correlated to \texttt{Churn\_Label}.

\begin{table}[H]

\caption{\label{tab:chi2services}Independence $\chi^2$ test between churn and services information variables}
\centering
\begin{tabular}[t]{lrrrl}
\toprule
  & Statistic & Df & Critical Value & p-value\\
\midrule
\cellcolor{gray!6}{Internet\_Service} & \cellcolor{gray!6}{728.70} & \cellcolor{gray!6}{2} & \cellcolor{gray!6}{5.99} & \cellcolor{gray!6}{5.8e-159}\\
Online\_Security & 205.42 & 1 & 3.84 & 1.4e-46\\
\cellcolor{gray!6}{Tech\_Support} & \cellcolor{gray!6}{189.97} & \cellcolor{gray!6}{1} & \cellcolor{gray!6}{3.84} & \cellcolor{gray!6}{3.2e-43}\\
Online\_Backup & 47.25 & 1 & 3.84 & 6.3e-12\\
\cellcolor{gray!6}{Device\_Protection} & \cellcolor{gray!6}{30.50} & \cellcolor{gray!6}{1} & \cellcolor{gray!6}{3.84} & \cellcolor{gray!6}{3.3e-08}\\
\addlinespace
Streaming\_TV & 27.84 & 1 & 3.84 & 1.3e-07\\
\cellcolor{gray!6}{Streaming\_Movies} & \cellcolor{gray!6}{25.76} & \cellcolor{gray!6}{1} & \cellcolor{gray!6}{3.84} & \cellcolor{gray!6}{3.9e-07}\\
Phone\_Service & 0.87 & 1 & 3.84 & 3.5e-01\\
\cellcolor{gray!6}{Multiple\_Lines} & \cellcolor{gray!6}{0.87} & \cellcolor{gray!6}{1} & \cellcolor{gray!6}{3.84} & \cellcolor{gray!6}{3.5e-01}\\
\bottomrule
\end{tabular}
\end{table}

Figure \ref{fig:kmservices} illustrates the \(\chi^2\) tests' results by representing the Kaplan-Meier estimated survivor function related to customer lifetime according to treatment variables on services subscribed. On the one hand, there seems to be no significant difference in terms of survival whether the customer uses phone service or not. The same remark can be pointed out based on whether the client has multiple lines as \texttt{Phone\_Service} and \texttt{Multiple\_Lines} might be quite correlated. In contrast, huge survival time difference can be noticed between customers with online security and those without, as well as between those having subscribed to technical support and those who have not. Finally, not using Internet service appears to have a positive influence on customer lifetime.

\begin{figure}

{\centering \includegraphics[width=75in]{./imgs/services_plot} 

}

\caption{Kaplan-Meier survival function depending on services subscribed}\label{fig:kmservices}
\end{figure}

\hypertarget{customer-account-data}{%
\subsection*{Customer account data}\label{customer-account-data}}
\addcontentsline{toc}{subsection}{Customer account data}

Variables on customer account such as the payment method used and the type of contract between the TSP and the client can be rich in information on customer lifetime. Indeed, table \ref{tab:chi2custaccount} shows that churn status strongly depends on the three variables, \texttt{Contract} being the most linked to \texttt{Churn\_Label}.

\begin{table}[H]

\caption{\label{tab:chi2custaccount}Independence $\chi^2$ test between churn and customer account data}
\centering
\begin{tabular}[t]{lrrrl}
\toprule
  & Statistic & Df & Critical Value & p-value\\
\midrule
\cellcolor{gray!6}{Contract} & \cellcolor{gray!6}{1179.55} & \cellcolor{gray!6}{2} & \cellcolor{gray!6}{5.99} & \cellcolor{gray!6}{7.3e-257}\\
Payment\_Method & 645.43 & 3 & 7.81 & 1.4e-139\\
\cellcolor{gray!6}{Paperless\_Billing} & \cellcolor{gray!6}{256.87} & \cellcolor{gray!6}{1} & \cellcolor{gray!6}{3.84} & \cellcolor{gray!6}{8.2e-58}\\
\bottomrule
\end{tabular}
\end{table}

Figure \ref{fig:kmcustaccount} enriches the \(\chi^2\) tests' results as it draws survival curves for each treatment variable's categories. When the firm/client contract is type month-to-month, the estimated survivor function decreases far more than for one-year or two-year contracts. In other words, the churn hazard is higher when the contract is renewed each month. This result makes sense as the customer may decide to leave the \emph{portfolio} once the month has ended as they are not commited for one or two years. Furthermore, clients with paperless billing contracts are more prone to churn, just like those paying by electronic check. It can be deduce that the \emph{attrition} risk is higher when the payment method is simplified.

\begin{figure}

{\centering \includegraphics[width=50in]{./imgs/account_info_plot} 

}

\caption{Kaplan-Meier survival function depending on customer account information}\label{fig:kmcustaccount}
\end{figure}

\hypertarget{cltv-customer-lifetime-value}{%
\section{\texorpdfstring{\texttt{CLTV}: Customer Lifetime Value}{CLTV: Customer Lifetime Value}}\label{cltv-customer-lifetime-value}}

In the dataset, the \emph{value} of the fictional TSP's customers is measured by the discrete quantitative variable called \texttt{CLTV}. In this context, it is decided to draw histogram and density plots related to \texttt{CLTV} depending on the three types of explanatory variables. Anova tests are also implemented to verify whether \texttt{CLTV} has different values in the treatment variables' categories. Anova is a generalization of the Student's t test allowing to compare more than two groups. The test's statistic computes the ratio between variance between sample and variance within samples and is Fisher distributed. A low ratio indicates that there is no significant difference between the means of the samples being compared.

\hypertarget{demographic-data-1}{%
\subsection*{Demographic data}\label{demographic-data-1}}
\addcontentsline{toc}{subsection}{Demographic data}

Based on the Anova tests' results, \texttt{Partner} and \texttt{Dependents} seem to be statistically discriminant in terms of customer lifetime value which is not the case for \texttt{Gender} and \texttt{Senior\_Citizen}.

\begin{table}[H]

\caption{\label{tab:aovdemographics}Anova test between CLTV and demographic variables}
\centering
\begin{tabular}[t]{lrrrl}
\toprule
  & F statistic & Df1 & Df2 & p-value\\
\midrule
\cellcolor{gray!6}{Partner} & \cellcolor{gray!6}{139.76} & \cellcolor{gray!6}{1} & \cellcolor{gray!6}{7030} & \cellcolor{gray!6}{6e-32}\\
Dependents & 24.89 & 1 & 7030 & 6.2e-07\\
\cellcolor{gray!6}{Gender} & \cellcolor{gray!6}{0.39} & \cellcolor{gray!6}{1} & \cellcolor{gray!6}{7030} & \cellcolor{gray!6}{5.3e-01}\\
Senior\_Citizen & 0.09 & 1 & 7030 & 7.6e-01\\
\bottomrule
\end{tabular}
\end{table}

The figure below illustrates how different CLV is between customers with a partner and those without, as well as between those with children or other dependents and those without. To put it another way, having a partner in life of dependents tends to increase customer lifetime value as shown by the last two plots.

\begin{figure}

{\centering \includegraphics[width=50in]{./imgs/cltv_demographics_plots} 

}

\caption{Histogram and density plots of customer lifetime value depending on demographic information}\label{fig:cltvdemographics}
\end{figure}

\hypertarget{data-on-services-subscribed-1}{%
\subsection*{Data on services subscribed}\label{data-on-services-subscribed-1}}
\addcontentsline{toc}{subsection}{Data on services subscribed}

When one wants to identify factors influencing customer lifetime value, it is relevant to consider variables related to services subscribed by customers. The Anova tests' results indicate that \texttt{CLTV} has significant different values in each group of all services related variables, except for \texttt{Internet\_Service}. The most important difference can be noted for \texttt{Online\_Backup} and \texttt{Online\_Security} variables, whereas there is more homogeneity in the \texttt{Phone\_Service} groups.

\begin{table}[H]

\caption{\label{tab:aovservices}Anova test between CLTV and services information variables}
\centering
\begin{tabular}[t]{lrrrl}
\toprule
  & F statistic & Df1 & Df2 & p-value\\
\midrule
\cellcolor{gray!6}{Online\_Backup} & \cellcolor{gray!6}{138.57} & \cellcolor{gray!6}{1} & \cellcolor{gray!6}{7030} & \cellcolor{gray!6}{1.1e-31}\\
Online\_Security & 138.21 & 1 & 7030 & 1.3e-31\\
\cellcolor{gray!6}{Device\_Protection} & \cellcolor{gray!6}{105.23} & \cellcolor{gray!6}{1} & \cellcolor{gray!6}{7030} & \cellcolor{gray!6}{1.6e-24}\\
Tech\_Support & 101.85 & 1 & 7030 & 8.7e-24\\
\cellcolor{gray!6}{Streaming\_Movies} & \cellcolor{gray!6}{90.96} & \cellcolor{gray!6}{1} & \cellcolor{gray!6}{7030} & \cellcolor{gray!6}{2e-21}\\
\addlinespace
Streaming\_TV & 79.58 & 1 & 7030 & 5.8e-19\\
\cellcolor{gray!6}{Phone\_Service} & \cellcolor{gray!6}{3.65} & \cellcolor{gray!6}{1} & \cellcolor{gray!6}{7030} & \cellcolor{gray!6}{5.6e-02}\\
Multiple\_Lines & 3.65 & 1 & 7030 & 5.6e-02\\
\cellcolor{gray!6}{Internet\_Service} & \cellcolor{gray!6}{0.56} & \cellcolor{gray!6}{2} & \cellcolor{gray!6}{7029} & \cellcolor{gray!6}{5.7e-01}\\
\bottomrule
\end{tabular}
\end{table}

From figure \ref{fig:cltvservices} one can note that subscribing to additional services like having multiple lines, online security and backup, device protection or using the streaming movie service significantly enhance customer lifetime value. These variables may be interesting predictors of \texttt{CLTV} in regression models.

\begin{figure}

{\centering \includegraphics[width=75in]{./imgs/cltv_services_plots} 

}

\caption{Histogram and density plots of customer lifetime value depending on services subscribed}\label{fig:cltvservices}
\end{figure}

\hypertarget{customer-account-data-1}{%
\subsection*{Customer account data}\label{customer-account-data-1}}
\addcontentsline{toc}{subsection}{Customer account data}

Table \ref{tab:aovaccountinfo} depicts that \texttt{CLTV} is statistically different according to the type of contract and the payment method. However, paperless billing does not seem to influence customer lifetime value.

\begin{table}[H]

\caption{\label{tab:aovaccountinfo}Anova test between CLTV and customer account data}
\centering
\begin{tabular}[t]{lrrrl}
\toprule
  & F statistic & Df1 & Df2 & p-value\\
\midrule
\cellcolor{gray!6}{Contract} & \cellcolor{gray!6}{274.28} & \cellcolor{gray!6}{2} & \cellcolor{gray!6}{7029} & \cellcolor{gray!6}{2e-115}\\
Payment\_Method & 52.66 & 3 & 7028 & 1.2e-33\\
\cellcolor{gray!6}{Paperless\_Billing} & \cellcolor{gray!6}{0.77} & \cellcolor{gray!6}{1} & \cellcolor{gray!6}{7030} & \cellcolor{gray!6}{3.8e-01}\\
\bottomrule
\end{tabular}
\end{table}

The three plots below provide details to previous results as it can be noticed that customers paying by credit card of bank transfer have higher CLV than those paying by e-check of mailed check. Besides, clients enrolled in one-year or two-year contracts have greater \emph{value} to the firm than those who pay on a monthly basis.

\begin{figure}

{\centering \includegraphics[width=50in]{./imgs/cltv_account_info_plots} 

}

\caption{Histogram and density plots of customer lifetime value depending on customer account data}\label{fig:cltvaccountinfo}
\end{figure}

\hypertarget{correlation-between-cltv-and-explanatory-quantitative-variables}{%
\subsection*{\texorpdfstring{Correlation between \texttt{CLTV} and explanatory quantitative variables}{Correlation between CLTV and explanatory quantitative variables}}\label{correlation-between-cltv-and-explanatory-quantitative-variables}}
\addcontentsline{toc}{subsection}{Correlation between \texttt{CLTV} and explanatory quantitative variables}

Plotting the correlation matrix allows to have an overview on the links between the dataset's quantitative variables. The method used is the Pearson correlation coefficient which is defined as follows:

\begin{equation}
  \rho_{XY} = \frac{\text{cov}(X, Y)}{\sigma_X \sigma_Y}
  \label{eq:pearson}
\end{equation}

where \(X\) and \(Y\) are two quantitative random variables, \(\text{cov}\) the covariance function and \(\sigma\) the standard deviation.

On figure \ref{fig:corrplot}, non-significant correlations are crossed-out. It can be noticed that \texttt{CLTV} has the strongest correlation with \texttt{Tenure\_Months} (\(40\%\)), followed by \texttt{Total\_Charges} (\(34\%\)) then \texttt{Monthly\_Charges} (\(10\%\)).

\begin{figure}

{\centering \includegraphics[width=25in]{./imgs/corr_plot} 

}

\caption{Correlation plot}\label{fig:corrplot}
\end{figure}

\hypertarget{churn-duration-and-customer-value}{%
\section{\texorpdfstring{Churn, duration and customer \emph{value}}{Churn, duration and customer value}}\label{churn-duration-and-customer-value}}

The final step in the data exploration consists in analyzing the relationship between the three target variables: \texttt{CLTV}, \texttt{Churn\_Label} and \texttt{Tenure\_Months}.

Looking at figure \ref{fig:cltvchurn}, customer lifetime value seems to have higher values for retained customers than for churners as the density is more right-oriented. This result makes sense as retained customers may have longer lifetime leading to higher CLV.

\begin{figure}

{\centering \includegraphics[width=400pt,height=300pt]{./imgs/cltv_churn_plot} 

}

\caption{Customer lifetime value depending on churn status}\label{fig:cltvchurn}
\end{figure}

Besides, the low \emph{p value} related to the Anova test between \texttt{CLTV} and \texttt{Churn\_Label} indicates that customer lifetime value is statistically different between churner and retained clients.

\begin{table}[H]

\caption{\label{tab:aovcltvchurn}Anova test between CLTV and churn status}
\centering
\begin{tabular}[t]{lrrrl}
\toprule
  & F statistic & Df1 & Df2 & p-value\\
\midrule
\cellcolor{gray!6}{Churn\_Label} & \cellcolor{gray!6}{117.57} & \cellcolor{gray!6}{1} & \cellcolor{gray!6}{7030} & \cellcolor{gray!6}{3.5e-27}\\
\bottomrule
\end{tabular}
\end{table}

The following histograms are interesting to the extent that the distribution of \texttt{Tenure\_Months} depends on the churn status. From figure \ref{fig:churndur}, one can note an inflation of low and high values for retained customers. The distribution appears to be more homogeneous for retained clients than for churners. These lasts' tenure months distribution is decreasing and looks like a Poisson distribution with an inflation of low values.

\begin{figure}

{\centering \includegraphics[width=500pt,height=350pt]{./imgs/duration_churn_plot} 

}

\caption{Tenure months depending on churn status}\label{fig:churndur}
\end{figure}

Eventually, figure \ref{fig:cltvdur} depicts the average customer lifetime value per number of months in the portfolio. Two distinct groups can be identified with a threshold at 50 months. On average, the most valuable customers are those who have been in the portfolio for more than 50 months.

\begin{figure}

{\centering \includegraphics[width=450pt,height=300pt]{./imgs/avg_cltv_plot} 

}

\caption{Average CLTV depending on number of months in the portfolio}\label{fig:cltvdur}
\end{figure}

\hypertarget{estimation}{%
\chapter{Estimation techniques}\label{estimation}}

This chapter explains the different methods used to model a portfolio of customers as well as their related risk of attrition and the overall value of the portfolio. In a first part, clustering techniques are implemented to identify segments of customers based on services and account variables. Then, survival models are fitted to estimate each customer's survival in the firm's portfolio. The selected model can also be used to assess the effects of each variable on the risk of churn. Finally, we answer the study's problematic by computing an estimated value of the portfolio. The latter is calculated using a corporate formula and takes customers' monthly fees and survival probabilities as inputs. The estimated portfolio value is not cost-adjusted there is no information on consumers' costs in the data set.

\hypertarget{featureselection}{%
\section{Feature selection}\label{featureselection}}

Before fitting any survival model or clustering algorithm to the data, the initial step consists in selecting variables that are discriminating in terms of churn hazard. Based on Kaplan-Meier analysis depicted in section \ref{churndescstats}, we have a general overview of features which influence the survival probability. In other words, our feature selection method relies on results obtained with descriptive statistics.

Table \ref{tab:selectedfeatures} shows the selected variables for 5 random observations extracted from the data set. It can be noted that these features are related to account or service information, apart from \texttt{Dependents} which indicates whether the client lives with any dependents (children, parents, etc) and \texttt{Senior\_Citizen}. Furthermore, 9 out of the 10 selected variables are categorical which implies that the estimation results could be used to compare different groups of client. \texttt{Monthly\_Charges} is the only quantitative variable used to fit clustering models and survival regressions.

\begin{table}[H]

\caption{\label{tab:selectedfeatures}Explanatory variables used in survival models and cluster analysis}
\centering
\begin{tabular}[t]{lllllllllr}
\toprule
Senior\_Citizen & Dependents & Phone\_Service & Internet\_Service & Online\_Security & Online\_Backup & Tech\_Support & Contract & Payment\_Method & Monthly\_Charges\\
\midrule
\cellcolor{gray!6}{No} & \cellcolor{gray!6}{No} & \cellcolor{gray!6}{Yes} & \cellcolor{gray!6}{Fiber optic} & \cellcolor{gray!6}{No} & \cellcolor{gray!6}{No} & \cellcolor{gray!6}{No} & \cellcolor{gray!6}{Month-to-month} & \cellcolor{gray!6}{Electronic check} & \cellcolor{gray!6}{96.05}\\
No & No & Yes & No & No & No & No & Month-to-month & Mailed check & 20.40\\
\cellcolor{gray!6}{No} & \cellcolor{gray!6}{Yes} & \cellcolor{gray!6}{Yes} & \cellcolor{gray!6}{No} & \cellcolor{gray!6}{No} & \cellcolor{gray!6}{No} & \cellcolor{gray!6}{No} & \cellcolor{gray!6}{Two year} & \cellcolor{gray!6}{Mailed check} & \cellcolor{gray!6}{19.75}\\
No & No & Yes & Fiber optic & Yes & No & No & Month-to-month & Mailed check & 75.00\\
\cellcolor{gray!6}{No} & \cellcolor{gray!6}{No} & \cellcolor{gray!6}{Yes} & \cellcolor{gray!6}{Fiber optic} & \cellcolor{gray!6}{No} & \cellcolor{gray!6}{No} & \cellcolor{gray!6}{No} & \cellcolor{gray!6}{Month-to-month} & \cellcolor{gray!6}{Electronic check} & \cellcolor{gray!6}{80.85}\\
\bottomrule
\end{tabular}
\end{table}

\hypertarget{portfolio-segmentation}{%
\section{Portfolio segmentation}\label{portfolio-segmentation}}

Customer segmentation helps decision makers having a better understanding of their clients. It can then be used to enhance marketing strategies via personalization. In other words, segmentation can lead to target customers with offers and incentives personalized to their wants, needs and preferences. In order to make segmentation more accurate, it is more appropriate to use cluster analysis than predetermined thresholds or rules, even more when we have several variables at our disposal. In this context, this section focuses on applying clustering methods on features displayed in table \ref{tab:selectedfeatures}, apart from \texttt{Monthly\_Charges}.

\hypertarget{transforming-qualitative-variables-into-principal-axes}{%
\subsection{Transforming qualitative variables into principal axes}\label{transforming-qualitative-variables-into-principal-axes}}

The variables selected to perform cluster analysis being categorical, it is needed to transform them into continuous features. To that end, multiple correspondence analysis (MCA) is performed. MCA is a dimension reducing method which takes multiple categorical variables and seeks to identify associations between levels of those variables. MCA aims at highlighting features that separate classes of individuals, while determining links between variables and categories. To that end, MCA keeps the core information by the means of principal components which are projected axes \citep{MCA}.

Here, the main objective of applying MCA being to obtain continuous features, it is decided to keep as many axes as it takes to have at least 80\% cumulated variance. In other words, we want the principal components to gather enough customer-related information. After having processed the \texttt{MCA} function from the R package \texttt{FactoMineR} \citep{FactoMineR2008}, 10 principal components are required to keep more than 80\% cumulated variance as depicted by figure \ref{fig:mcascreeplot}.

\begin{figure}

{\centering \includegraphics[width=15.14in]{./imgs/mca_screeplot} 

}

\caption{Variance explained and cumulated variance after MCA}\label{fig:mcascreeplot}
\end{figure}

Now the 10 continuous axes are identified, the next step consists in retrieving the customers' coordinates onto those axes to then perform cluster analysis.

\begin{table}[H]

\caption{\label{tab:unnamed-chunk-26}The 10 principal axes obtained by MCA}
\centering
\begin{tabular}[t]{rrrrrrrrrrr}
\toprule
CustomerID & Dim 1 & Dim 2 & Dim 3 & Dim 4 & Dim 5 & Dim 6 & Dim 7 & Dim 8 & Dim 9 & Dim 10\\
\midrule
\cellcolor{gray!6}{854} & \cellcolor{gray!6}{-0.13} & \cellcolor{gray!6}{-0.44} & \cellcolor{gray!6}{-0.29} & \cellcolor{gray!6}{-0.20} & \cellcolor{gray!6}{-0.02} & \cellcolor{gray!6}{0.00} & \cellcolor{gray!6}{0.25} & \cellcolor{gray!6}{-0.15} & \cellcolor{gray!6}{-0.14} & \cellcolor{gray!6}{-0.07}\\
1063 & -0.33 & -0.37 & -0.45 & -0.20 & -0.03 & -0.03 & 0.16 & -0.26 & -0.13 & 0.16\\
\cellcolor{gray!6}{2454} & \cellcolor{gray!6}{0.59} & \cellcolor{gray!6}{0.37} & \cellcolor{gray!6}{0.09} & \cellcolor{gray!6}{-0.16} & \cellcolor{gray!6}{-0.47} & \cellcolor{gray!6}{0.11} & \cellcolor{gray!6}{-0.07} & \cellcolor{gray!6}{-0.39} & \cellcolor{gray!6}{0.24} & \cellcolor{gray!6}{-0.06}\\
364 & 0.01 & -0.88 & 0.08 & -0.04 & -0.04 & 0.21 & -0.02 & 0.25 & 0.03 & 0.24\\
\cellcolor{gray!6}{3606} & \cellcolor{gray!6}{-0.65} & \cellcolor{gray!6}{-0.02} & \cellcolor{gray!6}{0.05} & \cellcolor{gray!6}{0.58} & \cellcolor{gray!6}{0.08} & \cellcolor{gray!6}{-0.13} & \cellcolor{gray!6}{-0.03} & \cellcolor{gray!6}{0.01} & \cellcolor{gray!6}{-0.38} & \cellcolor{gray!6}{-0.09}\\
\bottomrule
\end{tabular}
\end{table}

Note that a more in-depth visualisation of those 10 principal components can be found in section \ref{mcaappendix} in the appendix. This charts depict the percent contribution of each variable's categories to the principal axes, which is helpful to have a better understanding of MCA results.

\hypertarget{hierarchical-clustering-on-principal-components}{%
\subsection{Hierarchical clustering on principal components}\label{hierarchical-clustering-on-principal-components}}

Multiple correspondence analysis has led us to convert the categorical variables related to account and services information into 10 numerical projected axes. The stake here is to use the customers' projections onto the MCA components in order to identify groups of individuals through clustering techniques. As a reminder, clusters are expected to discriminate between customers based on the services they use and the type of plan they are enrolled into. The method implemented in this part relies on hierarchical clustering on principal components (HCPC).

\hypertarget{optimal-number-of-clusters}{%
\subsubsection*{Optimal number of clusters}\label{optimal-number-of-clusters}}
\addcontentsline{toc}{subsubsection}{Optimal number of clusters}

The key parameter to optimize when applying clustering methods is the number of clusters \(k\). When using the \texttt{HCPC} function from \texttt{FactorMineR}, the \texttt{nb.clust} parameter is set to -1 so that the tree is automatically cut at the suggested level. More precisely, the function first builds a hierarchical tree. Then the sum of the between-cluster inertia is calculated for each partition. The suggested partition is the one with the higher relative gain in inertia. Intuitively, the underlying objective is to choose a number of clusters leading to \(k\) well distinguished groups. Here, the between-cluster inertia is the metric measuring the amount of variability between clusters.

\begin{figure}

{\centering \includegraphics[width=13.89in]{./imgs/btw_inertia} 

}

\caption{Relative gains in between-cluster inertia given the partition}\label{fig:btwinertia}
\end{figure}

\hypertarget{cluster-visualisation} & \cellcolor{gray!6}{26.55} & \cellcolor{gray!6}{41.38} & \cellcolor{gray!6}{32.07}\\
\bottomrule
\end{tabular}
\end{table}

\begin{figure}

{\centering \includegraphics[width=12.83in]{./imgs/cluster_plot} 

}

\caption{Cluster visualisation onto the 2 first MCA axes}\label{fig:clusterplot}
\end{figure}

See figure \ref{fig:clusterplots} in the appendix for cluster visualisations onto the other MCA principal axes.

\hypertarget{cluster-description}{%
\subsubsection*{Cluster description}\label{cluster-description}}
\addcontentsline{toc}{subsubsection}{Cluster description}

\begin{figure}

{\centering \includegraphics[width=12.5in]{./imgs/monthly_charges_clust} 

}

\caption{Monthly charges repartition across clusters}\label{fig:monthlychargesclust}
\end{figure}

\begin{figure}

{\centering \includegraphics[width=20.83in]{./imgs/cat_vars_clust} 

}

\caption{Customer categorical features based on cluster}\label{fig:catvarsclust}
\end{figure}

\hypertarget{churn-analysis}{%
\section{Churn analysis}\label{churn-analysis}}

Estimating the risk of attrition related to each customer is an essential step to model the firm's portfolio. In this context, survival models can be implemented with a view of deriving a predicted churn risk and survival function for each client. One the one hand, these predictions can be used to identify loyal consumers and make appropriate decisions. For instance, it might be relevant to offer benefits to a high-value client with a high estimated churn risk. On the other hand, a customer's survival probability at time \(t\) represents the chance that this very customer be active in the portfolio at time \(t\). This measure is helpful to compute the estimated value of the portfolio in the last section.

Before presenting the estimation results, it seems important to recall that \texttt{Tenure\_Months} and \texttt{Churn\_Value} can be seen as a pair of time and event variables used as target in survival models.

\begin{table}[H]

\caption{\label{tab:survtarget}Time and event variables for survival models}
\centering
\begin{tabular}[t]{rrr}
\toprule
CustomerID & Tenure\_Months & Churn\_Value\\
\midrule
\cellcolor{gray!6}{2058} & \cellcolor{gray!6}{41} & \cellcolor{gray!6}{0}\\
6187 & 34 & 0\\
\cellcolor{gray!6}{4576} & \cellcolor{gray!6}{44} & \cellcolor{gray!6}{0}\\
1564 & 47 & 1\\
\cellcolor{gray!6}{4710} & \cellcolor{gray!6}{19} & \cellcolor{gray!6}{0}\\
\bottomrule
\end{tabular}
\end{table}

\hypertarget{the-cox-model}{%
\subsection{The Cox model}\label{the-cox-model}}

When it comes to choose an estimation method on survival data, the Cox PH model appears to be an interesting first choice. As explained in chapter \ref{duration}, this semi-parametric model makes no assumption regarding the nature of the baseline hazard function \(\lambda_0(t)\). The parametric part only relies in the modelling of the effect of some covariates on the hazard function \(\lambda(t)\) (see section \ref{coxph} for more details).

\hypertarget{fitting-the-model-on-the-selected-features}{%
\subsubsection*{Fitting the model on the selected features}\label{fitting-the-model-on-the-selected-features}}
\addcontentsline{toc}{subsubsection}{Fitting the model on the selected features}

Using the \texttt{coxph} function from the \texttt{survival} R library \citep{survival-book}, we are able to train a Cox model on the feature vector identified in section \ref{featureselection}. Once the model fitted, it seems relevant to evaluate its performance on the train data set. Table \ref{tab:lrtest} compares the model's log-likelihood to the constrained model's. Given the very low p-value, it can be assumed that the Cox model better fits the data than a model with only the intercept.

\begin{table}[H]

\caption{\label{tab:lrtest}Log-likelihood ratio test}
\centering
\begin{tabular}[t]{lrrr}
\toprule
  & Model & Constrained & pvalue\\
\midrule
\cellcolor{gray!6}{} & \cellcolor{gray!6}{-9228.77} & \cellcolor{gray!6}{-10448.08} & \cellcolor{gray!6}{0}\\
\bottomrule
\end{tabular}
\end{table}

Concordance index \(c\) is another metric to assess the performance of models which produces risk scores and is defined as the probability to well predict the order of event occurring time for any pair of instances. For the Cox model, the C-index obtained on the training set is \(c \approx 0.865 \pm 0.004\), which is more than satisfying.

\hypertarget{marginal-effects-1}{%
\subsubsection*{Marginal effects}\label{marginal-effects-1}}
\addcontentsline{toc}{subsubsection}{Marginal effects}

In the Cox model, the relative hazard between two observations is assumed to be constant over time. As a consequence, the relative hazard becomes \(\exp \hat{\beta}\) for both dummy and continuous variables. For instance regarding figure \ref{fig:coxmarginaleffects}, the relative hazard ratio between customers with a two-year contract and those with a month-to-month contract is 0.046, meaning that the latter group is 22 times more prone to churn than the former. Also, month-to-month clients are about 5 times more likely to churn than customers enrolled in one-year plan.

\begin{figure}

{\centering \includegraphics[width=18.06in]{./imgs/cox_ggforest} 

}

\caption{Marginal effects obtained with the Cox PH model}\label{fig:coxmarginaleffects}
\end{figure}

\hypertarget{estimating-churn-hazard}{%
\subsubsection*{Estimating churn hazard}\label{estimating-churn-hazard}}
\addcontentsline{toc}{subsubsection}{Estimating churn hazard}

\begin{figure}

{\centering \includegraphics[width=20.83in]{./imgs/cox_data_viz} 

}

\caption{Aggregated churn hazard, survival and cumulative hazard functions estimated by Cox model}\label{fig:coxdataviz}
\end{figure}

\begin{figure}

{\centering \includegraphics[width=20.83in]{./imgs/churn_surv_clust} 

}

\caption{Aggregated churn hazard, survival and cumulative hazard functions for each cluster}\label{fig:coxclust}
\end{figure}

\hypertarget{model-comparison}{%
\subsection{Model comparison}\label{model-comparison}}

\hypertarget{portfolio-value-estimation}{%
\section{Portfolio value estimation}\label{portfolio-value-estimation}}

\hypertarget{mathematical-formulation}{%
\subsection{Mathematical formulation}\label{mathematical-formulation}}

\hypertarget{customer-lifetime-raw-value}{%
\subsection{Customer lifetime raw value}\label{customer-lifetime-raw-value}}

\begin{table}[H]

\caption{\label{tab:custValues}CLRV Statistical summary}
\centering
\begin{tabular}[t]{lrrrrrr}
\toprule
  & Min. & 1st Qu. & Median & Mean & 3rd Qu. & Max.\\
\midrule
\cellcolor{gray!6}{} & \cellcolor{gray!6}{140.41} & \cellcolor{gray!6}{731.61} & \cellcolor{gray!6}{1411.27} & \cellcolor{gray!6}{2286.52} & \cellcolor{gray!6}{3752.24} & \cellcolor{gray!6}{6825.71}\\
\bottomrule
\end{tabular}
\end{table}

\begin{figure}

{\centering \includegraphics[width=19.44in]{./imgs/cust_lifetime_raw_value} 

}

\caption{Customer lifetime raw value - Density and evolution by number of months}\label{fig:custrawvaldens}
\end{figure}

\hypertarget{cluster-contribution-to-the-portfolio-value}{%
\subsection{Cluster contribution to the portfolio value}\label{cluster-contribution-to-the-portfolio-value}}

\begin{table}[H]

\caption{\label{tab:totValclust}Total CLRV per cluster}
\centering
\begin{tabular}[t]{ccccc}
\toprule
Cluster & Proportion (\%) & V lower & V & V upper\\
\midrule
\cellcolor{gray!6}{1} & \cellcolor{gray!6}{26.55} & \cellcolor{gray!6}{1,748,207} & \cellcolor{gray!6}{1,809,489} & \cellcolor{gray!6}{1,889,476}\\
2 & 41.38 & 4,666,325 & 4,875,438 & 5,136,889\\
\cellcolor{gray!6}{3} & \cellcolor{gray!6}{32.07} & \cellcolor{gray!6}{9,282,801} & \cellcolor{gray!6}{9,393,895} & \cellcolor{gray!6}{9,535,978}\\
\bottomrule
\end{tabular}
\end{table}

\begin{table}[H]

\caption{\label{tab:custValuesclust}CLRV statistical summary for each customer segment}
\centering
\begin{tabular}[t]{lrrrrrr}
\toprule
  & Min. & 1st Qu. & Median & Mean & 3rd Qu. & Max.\\
\midrule
\cellcolor{gray!6}{Cluster 1} & \cellcolor{gray!6}{141.10} & \cellcolor{gray!6}{653.93} & \cellcolor{gray!6}{930.95} & \cellcolor{gray!6}{969.20} & \cellcolor{gray!6}{1138.39} & \cellcolor{gray!6}{4727.19}\\
Cluster 2 & 140.41 & 459.49 & 1027.12 & 1675.41 & 2435.33 & 6425.49\\
\cellcolor{gray!6}{Cluster 3} & \cellcolor{gray!6}{384.60} & \cellcolor{gray!6}{2876.69} & \cellcolor{gray!6}{4312.25} & \cellcolor{gray!6}{4165.81} & \cellcolor{gray!6}{5576.70} & \cellcolor{gray!6}{6825.71}\\
\bottomrule
\end{tabular}
\end{table}

\hypertarget{simulations}{%
\subsection{Simulations}\label{simulations}}

\hypertarget{influence-of-the-discount-rate-on-clrv} & \cellcolor{gray!6}{20,894,006} & \cellcolor{gray!6}{21,721,297} & \cellcolor{gray!6}{22,674,662}\\
2\% & 20,358,160 & 21,158,861 & 22,080,569\\
\cellcolor{gray!6}{4\%} & \cellcolor{gray!6}{19,357,495} & \cellcolor{gray!6}{20,108,832} & \cellcolor{gray!6}{20,971,852}\\
8\% & 15,697,334 & 16,078,822 & 16,562,344\\
\bottomrule
\end{tabular}
\end{table}

\hypertarget{appendix}{%
\chapter*{Appendix}\label{appendix}}
\addcontentsline{toc}{chapter}{Appendix}

In this section some proofs of the mathematical concepts used in the study are derived, specifically related to duration analysis. It also consists of additional data visualisation related to the chapter \ref{estimation}.

\hypertarget{hazard-function}{%
\section*{Hazard function}\label{hazard-function}}
\addcontentsline{toc}{section}{Hazard function}

\begin{equation}    
  \begin{aligned}
  \lambda(t) & = \lim_{\Delta t \to 0} \frac{P\big[t \leq T < t + \Delta t | T \geq t \big]}{\Delta t} \\\\
  & = \lim_{\Delta t \to 0} \frac{P\big[t \leq T < t + \Delta t \big] / P\big[T \geq t  \big]}{\Delta t} \\\\
  & = \lim_{\Delta t \to 0} \frac{\big(F(t+\Delta t)-F(t)\big) / \Delta t}{S(t)} \\\\
  & = \frac{\text{d} F(t) / \text{d} t}{S(t)} \\\\
  & = \frac{f(t)}{S(t)} \\\\
  & = - \frac{\text{d}S(t) / \text{d} t}{S(t)} \\\\
  \lambda(t) & = \frac{-\text{d} \ln \big(S(t)\big)}{\text{d} t}
  \end{aligned}
  \label{eq:hazfunproof}
\end{equation}

\hypertarget{link-between-cumulative-hazard-and-survivor-functions}{%
\section*{Link between cumulative hazard and survivor functions}\label{link-between-cumulative-hazard-and-survivor-functions}}
\addcontentsline{toc}{section}{Link between cumulative hazard and survivor functions}

\begin{equation}
  \begin{aligned}
       & \Lambda(t) = \int_{0}^{t} \lambda(s)ds \\\\
  \iff & \Lambda(t) = \int_{0}^{t} \frac{f(s)}{S(s)}ds \\\\
  \iff & \Lambda(t) = -\ln \big(S(t)\big) \\\\
  \iff & S(t) = \exp \big(-\Lambda(t)\big)
  \end{aligned}
  \label{eq:linksurvcumhaz}
\end{equation}

\hypertarget{contribution-to-the-partial-likelihood-function-in-ph-models}{%
\section*{Contribution to the partial likelihood function in PH models}\label{contribution-to-the-partial-likelihood-function-in-ph-models}}
\addcontentsline{toc}{section}{Contribution to the partial likelihood function in PH models}

\begin{equation}
\begin{aligned}
  \mathbb{P}\big[T_j = t_j | R(t_j) \big] & = \frac{\mathbb{P}\big[T_j = t_j | T_j \geq t_j \big]}{\sum_{l \in R(t_j)} \ \mathbb{P}\big[T_l = t_l | T_l \geq t_j \big]} \\\\
  & = \frac{\lambda_j(t_j|\mathrm{x_j}, \beta)}{\sum_{l \in R(t_j)} \ \lambda_l(t_l|\mathrm{x_l}, \beta)} \\\\
  & = \frac{\lambda_0 (t_j, \alpha)\phi(\mathrm{x_j}, \beta)}{\sum_{l \in R(t_j)} \ \lambda_0 (t_j, \alpha)\phi(\mathrm{x_l}, \beta)} \\\\
  \mathbb{P}\big[T_j = t_j | R(t_j) \big] & = \frac{\phi(\mathrm{x_j}, \beta)}{\sum_{l \in R(t_j)} \phi(\mathrm{x_l}, \beta)}
\end{aligned}
\label{eq:contribpartiallikproof}
\end{equation}

\hypertarget{partial-likelihood-function-in-ph-models}{%
\section*{Partial likelihood function in PH models}\label{partial-likelihood-function-in-ph-models}}
\addcontentsline{toc}{section}{Partial likelihood function in PH models}

Based on equation \eqref{eq:contribpartiallik}, one can derive the probability that all spells completed at \(t_j\) ends in the \(j^{\text{th}}\) failure time, such that:

\begin{equation}
\begin{aligned}
  \mathcal{L}_{p,\ t_j} & = \mathbb{P}\big[T_1 = t_j, \dots, T_{d_j} = t_j \ | \ R(t_j)\big] \\\\
  & = \Pi_{m \in D(t_j)} \ \mathbb{P}\big[T_m = t_j | R(t_j) \big] \\\\
  & = \Pi_{m \in D(t_j)} \ \frac{\phi(\mathrm{x_m}, \beta)}{\sum_{l \in R(t_j)} \phi(\mathrm{x_l}, \beta)} \\\\
  & = \Pi_{m \in D(t_j)} \ \phi(\mathrm{x_m}, \beta) \times \Pi_{m \in D(t_j)} \ \frac{1}{\sum_{l \in R(t_j)} \phi(\mathrm{x_l}, \beta)} \\\\
  \mathcal{L}_{p,\ t_j} & = \frac{\Pi_{m \in D(t_j)} \ \phi(\mathrm{x_m}, \beta)}{\Big[\sum_{l \in R(t_j)} \phi(\mathrm{x_l}, \beta)\Big]^{d_j}}
\end{aligned}
\label{eq:partlikproof}
\end{equation}

The joint probability over the \(k\) ordered discrete failure times then becomes:

\begin{equation}
\begin{aligned}
  \mathcal{L}_p & = \Pi_{j=1}^{k} \ \mathcal{L}_{p,\ t_j} \\\\
  \mathcal{L}_p & = \Pi_{j=1}^{k} \ \frac{\Pi_{m \in D(t_j)} \ \phi(\mathrm{x_m}, \beta)}{\Big[\sum_{l \in R(t_j)} \phi(\mathrm{x_l}, \beta)\Big]^{d_j}}
\end{aligned}
\label{eq:partlikproofbis}
\end{equation}

\hypertarget{mcaappendix}{%
\section*{Multiple correspondence analysis}\label{mcaappendix}}
\addcontentsline{toc}{section}{Multiple correspondence analysis}

\begin{figure}

{\centering \includegraphics[width=15.28in]{./imgs/mca_contrib_12} 

}

\caption{MCA -  Categories contribution to axes 1 and 2}\label{fig:mcacontrib12}
\end{figure}

\begin{figure}

{\centering \includegraphics[width=15.28in]{./imgs/mca_contrib_34} 

}

\caption{MCA -  Categories contribution to axes 3 and 4}\label{fig:mcacontrib34}
\end{figure}

\begin{figure}

{\centering \includegraphics[width=15.28in]{./imgs/mca_contrib_56} 

}

\caption{MCA -  Categories contribution to axes 5 and 6}\label{fig:mcacontrib56}
\end{figure}

\begin{figure}

{\centering \includegraphics[width=15.28in]{./imgs/mca_contrib_78} 

}

\caption{MCA -  Categories contribution to axes 7 and 8}\label{fig:mcacontrib78}
\end{figure}

\begin{figure}

{\centering \includegraphics[width=15.28in]{./imgs/mca_contrib_910} 

}

\caption{MCA -  Categories contribution to axes 9 and 10}\label{fig:mcacontrib910}
\end{figure}

\hypertarget{hierarchical-clustering-on-principal-components-1}{%
\section*{Hierarchical clustering on principal components}\label{hierarchical-clustering-on-principal-components-1}}
\addcontentsline{toc}{section}{Hierarchical clustering on principal components}

\hypertarget{cluster-visualisation-1}{%
\subsection*{Cluster visualisation}\label{cluster-visualisation-1}}
\addcontentsline{toc}{subsection}{Cluster visualisation}

\begin{figure}

{\centering \includegraphics[width=15.28in]{./imgs/cluster_plots} 

}

\caption{Cluster visualisation onto MCA axes}\label{fig:clusterplots}
\end{figure}

\hypertarget{parangons}{%
\subsection*{Parangons}\label{parangons}}
\addcontentsline{toc}{subsection}{Parangons}

Data related to cluster 1's parangons

\begin{table}[H]

\caption{\label{tab:para1dat}}
\centering
\begin{tabular}[t]{lllllllllllllrrrr}
\toprule
  & Senior\_Citizen & Dependents & Internet\_Service & Online\_Security & Online\_Backup & Device\_Protection & Tech\_Support & Streaming\_TV & Streaming\_Movies & Contract & Paperless\_Billing & Payment\_Method & Monthly\_Charges & Tenure\_Months & Churn\_Value & CLTV\\
\midrule
\cellcolor{gray!6}{8} & \cellcolor{gray!6}{No} & \cellcolor{gray!6}{No} & \cellcolor{gray!6}{No} & \cellcolor{gray!6}{No} & \cellcolor{gray!6}{No} & \cellcolor{gray!6}{No} & \cellcolor{gray!6}{No} & \cellcolor{gray!6}{No} & \cellcolor{gray!6}{No} & \cellcolor{gray!6}{Month-to-month} & \cellcolor{gray!6}{No} & \cellcolor{gray!6}{Mailed check} & \cellcolor{gray!6}{20.15} & \cellcolor{gray!6}{1} & \cellcolor{gray!6}{1} & \cellcolor{gray!6}{4832}\\
23 & No & No & No & No & No & No & No & No & No & Month-to-month & No & Mailed check & 21.05 & 5 & 1 & 2604\\
\cellcolor{gray!6}{144} & \cellcolor{gray!6}{No} & \cellcolor{gray!6}{No} & \cellcolor{gray!6}{No} & \cellcolor{gray!6}{No} & \cellcolor{gray!6}{No} & \cellcolor{gray!6}{No} & \cellcolor{gray!6}{No} & \cellcolor{gray!6}{No} & \cellcolor{gray!6}{No} & \cellcolor{gray!6}{Month-to-month} & \cellcolor{gray!6}{No} & \cellcolor{gray!6}{Mailed check} & \cellcolor{gray!6}{19.00} & \cellcolor{gray!6}{12} & \cellcolor{gray!6}{1} & \cellcolor{gray!6}{3535}\\
178 & No & No & No & No & No & No & No & No & No & Month-to-month & No & Mailed check & 19.55 & 1 & 1 & 3556\\
\cellcolor{gray!6}{190} & \cellcolor{gray!6}{No} & \cellcolor{gray!6}{No} & \cellcolor{gray!6}{No} & \cellcolor{gray!6}{No} & \cellcolor{gray!6}{No} & \cellcolor{gray!6}{No} & \cellcolor{gray!6}{No} & \cellcolor{gray!6}{No} & \cellcolor{gray!6}{No} & \cellcolor{gray!6}{Month-to-month} & \cellcolor{gray!6}{No} & \cellcolor{gray!6}{Mailed check} & \cellcolor{gray!6}{19.90} & \cellcolor{gray!6}{1} & \cellcolor{gray!6}{1} & \cellcolor{gray!6}{4248}\\
\bottomrule
\end{tabular}
\end{table}

Data related to cluster 2's parangons

\begin{table}[H]

\caption{\label{tab:para2dat}}
\centering
\begin{tabular}[t]{lllllllllllllrrrr}
\toprule
  & Senior\_Citizen & Dependents & Internet\_Service & Online\_Security & Online\_Backup & Device\_Protection & Tech\_Support & Streaming\_TV & Streaming\_Movies & Contract & Paperless\_Billing & Payment\_Method & Monthly\_Charges & Tenure\_Months & Churn\_Value & CLTV\\
\midrule
\cellcolor{gray!6}{1405} & \cellcolor{gray!6}{No} & \cellcolor{gray!6}{No} & \cellcolor{gray!6}{Fiber optic} & \cellcolor{gray!6}{Yes} & \cellcolor{gray!6}{No} & \cellcolor{gray!6}{Yes} & \cellcolor{gray!6}{No} & \cellcolor{gray!6}{No} & \cellcolor{gray!6}{No} & \cellcolor{gray!6}{Month-to-month} & \cellcolor{gray!6}{Yes} & \cellcolor{gray!6}{Electronic check} & \cellcolor{gray!6}{84.50} & \cellcolor{gray!6}{5} & \cellcolor{gray!6}{1} & \cellcolor{gray!6}{5141}\\
2427 & No & No & Fiber optic & Yes & No & Yes & No & No & No & Month-to-month & Yes & Electronic check & 80.45 & 2 & 0 & 5602\\
\cellcolor{gray!6}{4455} & \cellcolor{gray!6}{No} & \cellcolor{gray!6}{No} & \cellcolor{gray!6}{Fiber optic} & \cellcolor{gray!6}{Yes} & \cellcolor{gray!6}{No} & \cellcolor{gray!6}{Yes} & \cellcolor{gray!6}{No} & \cellcolor{gray!6}{No} & \cellcolor{gray!6}{No} & \cellcolor{gray!6}{Month-to-month} & \cellcolor{gray!6}{Yes} & \cellcolor{gray!6}{Electronic check} & \cellcolor{gray!6}{84.75} & \cellcolor{gray!6}{19} & \cellcolor{gray!6}{0} & \cellcolor{gray!6}{5157}\\
5025 & No & No & Fiber optic & Yes & No & No & No & No & Yes & Month-to-month & Yes & Electronic check & 88.05 & 9 & 0 & 2715\\
\cellcolor{gray!6}{5743} & \cellcolor{gray!6}{No} & \cellcolor{gray!6}{No} & \cellcolor{gray!6}{Fiber optic} & \cellcolor{gray!6}{Yes} & \cellcolor{gray!6}{No} & \cellcolor{gray!6}{No} & \cellcolor{gray!6}{No} & \cellcolor{gray!6}{No} & \cellcolor{gray!6}{Yes} & \cellcolor{gray!6}{Month-to-month} & \cellcolor{gray!6}{Yes} & \cellcolor{gray!6}{Electronic check} & \cellcolor{gray!6}{90.65} & \cellcolor{gray!6}{33} & \cellcolor{gray!6}{0} & \cellcolor{gray!6}{4460}\\
\bottomrule
\end{tabular}
\end{table}

Data related to cluster 3's parangons

\begin{table}[H]

\caption{\label{tab:para3dat}}
\centering
\begin{tabular}[t]{lllllllllllllrrrr}
\toprule
  & Senior\_Citizen & Dependents & Internet\_Service & Online\_Security & Online\_Backup & Device\_Protection & Tech\_Support & Streaming\_TV & Streaming\_Movies & Contract & Paperless\_Billing & Payment\_Method & Monthly\_Charges & Tenure\_Months & Churn\_Value & CLTV\\
\midrule
\cellcolor{gray!6}{5350} & \cellcolor{gray!6}{No} & \cellcolor{gray!6}{No} & \cellcolor{gray!6}{DSL} & \cellcolor{gray!6}{Yes} & \cellcolor{gray!6}{Yes} & \cellcolor{gray!6}{Yes} & \cellcolor{gray!6}{No} & \cellcolor{gray!6}{No} & \cellcolor{gray!6}{Yes} & \cellcolor{gray!6}{Two year} & \cellcolor{gray!6}{Yes} & \cellcolor{gray!6}{Electronic check} & \cellcolor{gray!6}{49.20} & \cellcolor{gray!6}{72} & \cellcolor{gray!6}{0} & \cellcolor{gray!6}{4700}\\
3103 & No & No & DSL & Yes & Yes & Yes & No & Yes & Yes & Two year & Yes & Electronic check & 85.25 & 72 & 0 & 6275\\
\cellcolor{gray!6}{3936} & \cellcolor{gray!6}{No} & \cellcolor{gray!6}{No} & \cellcolor{gray!6}{DSL} & \cellcolor{gray!6}{Yes} & \cellcolor{gray!6}{Yes} & \cellcolor{gray!6}{Yes} & \cellcolor{gray!6}{No} & \cellcolor{gray!6}{Yes} & \cellcolor{gray!6}{Yes} & \cellcolor{gray!6}{Two year} & \cellcolor{gray!6}{Yes} & \cellcolor{gray!6}{Electronic check} & \cellcolor{gray!6}{86.10} & \cellcolor{gray!6}{58} & \cellcolor{gray!6}{0} & \cellcolor{gray!6}{5666}\\
1957 & No & No & DSL & No & Yes & No & Yes & Yes & Yes & Two year & Yes & Electronic check & 54.60 & 64 & 0 & 4850\\
\cellcolor{gray!6}{2999} & \cellcolor{gray!6}{No} & \cellcolor{gray!6}{No} & \cellcolor{gray!6}{DSL} & \cellcolor{gray!6}{No} & \cellcolor{gray!6}{Yes} & \cellcolor{gray!6}{No} & \cellcolor{gray!6}{Yes} & \cellcolor{gray!6}{Yes} & \cellcolor{gray!6}{Yes} & \cellcolor{gray!6}{Two year} & \cellcolor{gray!6}{Yes} & \cellcolor{gray!6}{Electronic check} & \cellcolor{gray!6}{78.95} & \cellcolor{gray!6}{72} & \cellcolor{gray!6}{0} & \cellcolor{gray!6}{5050}\\
\bottomrule
\end{tabular}
\end{table}

  \bibliography{biblio.bib,packages.bib}

\end{document}
